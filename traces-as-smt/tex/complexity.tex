\section{NP Completeness Proof} 
The complexity proof is inspired by the NP-completeness proof for
memory coherence and consistency by Cantin \emph{et al.} that uses a
similar reduction from SAT only in the context of shared memory
\cite{Cantin:2005:CVM:1070608.1070722}. The complexity proof is on a
new decision problem: \emph{Verifying Assertions in Message Passing}
(VAMP).
\begin{definition}
\textbf{Verifying Assertions in Message Passing}.
\begin{compactdesc}
\item{INSTANCE}: A set of constants $D$, a set of variables $X$, and a
  finite set $H$ of task histories consisting of send, receive, and
  assert operations over $X$ and $D$.
\item{QUESTION}: Is there a feasible schedule $S$ for the operations
  of $H$ that satisfy all the assertions?
\end{compactdesc}
\label{def:np1}
\end{definition}

The VAMP problem is NP-complete. The proof is a reduction from
SAT. Given an instance $Q$ of SAT consisting of a set of variables $U$
and set of clauses $C$ over $U$, an instance $V$ of VAMP is
constructed such that $Q$ has a feasible schedule $S$ that meets all
the assertions if and only if there is a satisfying truth assignment
for $Q$.

The reduction is illustrated in \figref{fig:vampi}. The figure elides
the explicit calls to wait which directly follow each send and receive
operation, and it elides the subscript notation as it is redundant in
the figure.

The reduction relies on non-determinism in the message passing to
decide the value of each variable in $U$.  The tasks $h_{d_0}$ and
$h_{d_1}$ repeatedly send the constant value $d_0$ (constant 0) or
$d_1$ (constant 1) to task $h_C$. The key intuition is that these
tasks are synchronized with $h_C$ so they essentially wait to send the
value until asked.

The task $h_C$ sequentially requests and receives $d_0$ and $d_1$
values for each variable in the SAT instance $Q$. It does not request
values for a new variable until the current variable is resolved. As
the values come from two separate tasks upon request, the messages
race in the runtime and may arrive in either order at $h_C$. As a
result, the value in each variable is non-deterministically $d_0$ or $d_1$.

After the values of each variable $u_i$ is resolved, the $h_C$ asserts
the truth of each value. As the clauses are conjunctive, the assertion
are sequentially evaluated. If a satisfying assignment exists for $Q$,
then a feasible schedule exists that resolves the values of each
variable in such a way that every assert holds.

\begin{figure}
\begin{center}
\setlength{\tabcolsep}{2pt}
\begin{tabular}[t]{|l|l|l|}
\hline
\multicolumn{3}{|l|}{\textbf{SAT:} $\mathit{U\equiv\{u_0,u_1,...,u_m\}}$}\\
              \multicolumn{3}{|l|}{$\mathit{C\equiv\{c_0,c_1,...,c_n\}}$}\\
              \multicolumn{3}{|l|}{$\mathit{Q\equiv\{c_0\wedge c_1\wedge ...\wedge c_n\}}$}\\
\hline
\multicolumn{3}{|l|}{\textbf{VAMPI:} $\mathit{H\equiv\{h_{d_0},h_{d_1},h_{C}\}}$}\\
                \multicolumn{3}{|l|}{$\mathit{X\equiv\{u_0,...,u_n,g_0,g_1\}}$}\\
                \multicolumn{3}{|l|}{$\mathit{D\equiv\{d_0,d_1\}}$}\\
\hline
$h_{d_0}$ & $h_{d_1}$ & $h_C$ \\
\hline
$R(g_{0},*)$   & $R(g_{1},*)$   & $S(d_0,h_{d_0})$ \\
$S(d_{0},h_C)$ & $S(d_{1},h_C)$ & $S(d_0,h_{d_1})$ \\
              &                & $R(u_0,*)$        \\
              &                & $R(u_0,*)$        \\
\hline
$R(g_{0},*)$  &  $R(g_{1},*)$   & $S(d_0,h_{d_0})$ \\
$S(d_{0},h_C)$&  $S(d_{1},h_C)$ & $S(d_0,h_{d_1})$ \\
              &                & $R(u_1,*)$       \\
              &                & $R(u_1,*)$       \\
\hline
$\ldots$      &  $\ldots$      & $\ldots$                \\
\hline
              &                & \textit{assert}($c_0$) \\
              &                & \textit{assert}($c_1$) \\
              &                & $\ldots$ \\
\hline
\end{tabular}
\end{center}
\caption{General SAT to VAMPI reduction}
\label{fig:vampi}
\end{figure}

\begin{lemma} \label{lem:sat}
$S$ is a feasible schedule for $H$ that satisfies all assertions if
  and only if $Q$ is satisfiable.
\end{lemma}
\begin{proof}
\textbf{V implies Q}: proof by contradiction. Assume that $Q$ is
unsatisfiable even though there is a feasible schedule $S$ for $V$
that meets all the assertions. The reduction in \figref{fig:vampi}
considers all truth values of the variables in $Q$, over every
combination, by virtue of the non-determinism, and then asserts the
truth of each of the clauses in $Q$. The complete set of possibilities
is realized by sending in parallel from $h_{d_0}$ and $h_{d_1}$ the
two truth valuations for a given variable to $h_C$. As these messages
may be received in any order, each variable may assume either truth
value. Further, each variable resolved is an independent choice so all
combinations of variable valuations must be considered.  This fact is
a contradiction to the assumption of $Q$ being unsatisfiable as the
same truth values that meet the assertions would be a satisfying
assignment in $Q$.

\noindent \textbf{Q implies V}: the proof is symmetric to the previous case in
proceeds in a like manner.
\end{proof}

\begin{theorem}[NP-complete]
VAMP is NP-complete.
\end{theorem}
\begin{proof}
\noindent\textbf{Membership in NP}: A certificate is a schedule
matching send and receives in each of the histories. The schedule is
linearly scanned with the histories and checked that it does not
violate MCAPI semantics. Our extended version constructs an
operational model of MCAPI semantics that does just such a check given
a schedule~\cite{extended-version}. The complexity is linear in the
size of the schedule.

\noindent\textbf{NP-hard}: Polynomial reduction from SAT. The
correctness of the reduction is established by \lemmaref{lem:sat}.
The remainder of the proof is the complexity of the reduction. There
are two tasks to send values $d_0$ and $d_1$ upon request. For each
variable $u_i \in U$, each of these tasks, $d_0$ and $d_1$, needs two
operations: one to synchronize with $h_C$ and another to send the
value: $O(2 * 2 * |U|)$. The task $h_C$ must request values from
$h_{d_0}$ and $h_{d_1}$ and then receive both those values, it must do
this for each variable: $O(2 * 2 * |U| + |C|)$.  Once all the values
are collected, it must them assert each clause: $O(|C|)$. As every
term is linear, the reduction is linear.
\end{proof}
