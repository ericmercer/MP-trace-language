\section{Conclusions and Future Work}
This paper presents a proof that the problem of resolving
non-determinism in message passing in a way that meets asserts is
NP-complete. The paper then presents an SMT encoding of an MCAPI
program execution that uses match pairs directly rather than the
state-based or order-based encoding in the prior work. The encoding is
generated from a given execution trace and a set of potential match
pairs that can be over-approximated. The encoding takes extra care in
forming the SMT problem to preclude bogus match pairs in any
over-approximation of the match pair input set. Critically, the
encoding is the first to correctly capture the non-deterministic
behaviors of an MCAPI program execution under infinite-buffer
semantics.

This paper further defines an algorithm with $O(N^2)$ time complexity
to over-approximate the true set of match pairs, where $N$ is the
total number of code lines of the program. A comparison to prior work,
\cite{elwakil:padtad10}, for a set of ``toy" examples under
zero-buffer semanics shows the new encoding capable and efficient in
capturing correct behaviors of an MCAPI program execution. Experiments
further show that the encoding scales to programs with significant levels of
non-determinism in how sends match to receives.

The results show that a large match-pair set does affect the runtime
performance of the encoding in the SMT problem even if the encoding is
sound under an over-approximation. Future work explores new methods
for generating a much more precise set of match pairs. The encoding is
dependent on an input execution trace of the program. Future work
explores integrating the encoding into a model checker. The model
checker generates a program trace that is encoded and verified. The
result is then used to inform the model checker as to where it needs
to backtrack to generate a new execution trace. The goal is to use the
trace verification to construct a better partial order reduction in
the model checker.

Finally, given the importance of high performance computing, future
work looks to extend the encoding to account for MPI collective
operations. This direction is motivated by the results where the
encoding seems to scale to significant levels of concurrency. It
should be possible to express MPI collectives as additional
constraints in the encoding and apply the technique to MPI programs
directly.
