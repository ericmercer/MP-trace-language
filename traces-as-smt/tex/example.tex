% ---------------------------------------------------------------------
% Save boxes for the various figures in the example section
% ---------------------------------------------------------------------
\newsavebox{\boxTZero}
\begin{lrbox}{\boxTZero}
\begin{minipage}[t]{0.65\linewidth}
\begin{alltt}
   00 initialize(NODE_0,&v,&s);
   01 ep0 = create_endpoint(PORT_0,&s);

04 02 msg_recv_i(ep0,A,sizeof(A),&rcvA,&s);
05 03 wait(&rcvA,&size,&s,MCAPI_INF);
06 04 a = atoi(A);

11 05 msg_recv_i(ep0,B,sizeof(B),&rcvB,&s);
12 06 wait(&rcvB,&size,&status,MCAPI_INF);
13 07 b = atoi(B);

14 08 if (b > 0)
15 09    assert(a == 4);

   10 finalize(&s);
\end{alltt}
\end{minipage}
\end{lrbox}

\newsavebox{\boxATZero}
\begin{lrbox}{\boxATZero}
\begin{minipage}[t]{0.65\linewidth}
\begin{alltt}
08 0_0        (0_0 (rcvi rcvA 0 a))
09 (0_1 (0 1)) (0_1 (wait rcvA))

10 0_2        (0_2 (rcvi rcvB 0 b))
11 (0_3 (0 2)) (0_3 (wait rcvB))

12 0_4        (0_4 (assume (> b 0)))
13 0_5        (0_5 (assert (= a 4)))
\end{alltt}
\end{minipage}
\end{lrbox}

\newsavebox{\boxTOne}
\begin{lrbox}{\boxTOne}
\begin{minipage}[t]{0.65\linewidth}
\begin{alltt}
   00 initialize(NODE_1,&v,&s);
   01 ep1 = create_endpoint(PORT_1,&s);
   02 ep0 = get_endpoint(NODE_0,PORT_0,&s);

07 03 msg_recv_i(ep1,C,sizeof(C),&rcvC,&s);
08 04 wait(&rcvC,&size,&s,INF);

09 05 msg_send_i(ep1,ep0,"1",2,N,&snd3,&s);
10 06 wait(&snd3,&size,&s,MCAPI_INF);

   07 finalize(&s);
\end{alltt}
\end{minipage}
\end{lrbox}

\newsavebox{\boxATOne}
\begin{lrbox}{\boxATOne}
\begin{minipage}[t]{0.65\linewidth}
\begin{alltt}
04 1_1        (1_1 (rcvi rcvC 1 c))
05 (1_2 (1 2)) (1_2 (wait rcvC))

06 1_3        (1_3 (sndi snd3 1 0 0x1))
07 1_4        (1_4 (wait snd3))
\end{alltt}
\end{minipage}
\end{lrbox}

\newsavebox{\boxTTwo}
\begin{lrbox}{\boxTTwo}
\begin{minipage}[t]{0.65\linewidth}
\begin{alltt}
   00 initialize(NODE_2,&v,&s);
   01 t2 = create_endpoint(PORT_2,&s);
   02 t0 = get_endpoint(NODE_0,PORT_0,&s);
   03 t1 = get_endpoint(NODE_1,PORT_1,&s);

00 04 msg_send_i(ep2,ep0,"4",2,N,&snd1,&s);
01 05 wait(&snd1,&size,&status,MCAPI_INF);

02 06 msg_send_i(ep2,ep1,"Go",3,N,&snd2,&s);
03 07 wait(&snd2,&size,&status,MCAPI_INF);
   08
   09 finalize(&s);
\end{alltt}
\end{minipage}
\end{lrbox}

\newsavebox{\boxATTwo}
\begin{lrbox}{\boxATTwo}
\begin{minipage}[t]{0.65\linewidth}
\begin{alltt}
00 2_0        (2_0 (sndi snd1 2 0 0x4))
01 2_1        (2_1 (wait snd1))

02 2_2        (2_2 (sndi snd2 2 1 0x476f00))
03 2_3        (2_3 (wait snd2))
\end{alltt}
\end{minipage}
\end{lrbox}

\newsavebox{\boxMP}
\begin{lrbox}{\boxMP}
\begin{minipage}[c]{0.2\linewidth}
\begin{alltt}
(rcvA snd1)
(rcvA snd3)

(rcvB snd1)
(rcvB snd3)

(rcvC snd2)
\end{alltt}
\end{minipage}
\end{lrbox}


\newsavebox{\boxSMT}
\begin{lrbox}{\boxSMT}
\begin{minipage}[c]{0.4\linewidth}
\begin{alltt}
(HB rcvA_loc wait_rcvA_loc)
(HB wait_rcvA_loc rcvB_loc)
(HB rcvB_loc wait_rcvB_loc)
(HB wait_rcvB_loc assume_loc)
(HB assume_loc assert_loc)
(HB rcvC_loc wait_rcvC_loc)
(HB wait_rcvC_loc snd3_loc)
(HB snd3_loc wait_snd3_loc)
(HB snd1_loc wait_snd1_loc)
(HB wait_snd1_loc snd2_loc)
(HB snd2_loc wait_snd2_loc)

(or (MATCH rcvA snd3)
    (MATCH rcvA snd1))
(or (MATCH rcvB  snd1)
    (MATCH rcvB snd3))
(ne rcvA rcvB)

(MATCH rcvC snd2)

(assume (> b 0))
(assert (not (= a 4)))
\end{alltt}
\end{minipage}
\end{lrbox}
% ---------------------------------------------------------------------
% END Save boxes
% ---------------------------------------------------------------------

\section{Example}

It is a challenge to explain intended behavior in simple scenarios
consisting of a handful of calls when dealing with concurrency. Consider
the MCAPI program in the top of \figref{fig:mcapi} that includes three
tasks that use send (\texttt{mcapi\_msg\_send\_i}) and receive
(\texttt{mcapi\_msg\_recv\_i}) calls to communicate with each other.
Line numbers appear in the second column for each task, and the
declarations of the local variables are omitted for space. Picking up
the scenario just after the endpoints are defined, lines \texttt{02}
and \texttt{05} of task 0 receive two messages on the endpoint
\textit{ep0} in variables $A$ and $B$ which are
converted to integer values and stored in variables $a$ and
$b$ on lines \texttt{04} and \texttt{07}; task 1 receives one
message on endpoint \textit{ep1} in variable \textit{C} on line
\texttt{03} and then sends the message \textit{``1''} on line \texttt{05} to
\textit{ep0}; and finally, task 2 sends messages \textit{``4''} and \textit{``Go''} on
lines \texttt{04} and \texttt{06} to endpoints \textit{ep0} and
\textit{ep1} respectively. Task 0 has additional code (lines \texttt{08} -
\texttt{10}) to assert properties of the values in $a$ and
$b$. The \texttt{mcapi\_wait} calls block until the associated
send or receive buffer is able to be used. Given the scenario, we
might ask the question: \emph{``What are the possible values of
\texttt{a} and \texttt{b} after the scenario completes?''}

Intuitively, variable \textit{a} should contain $4$ and variable
\textit{b} should contain $1$ since task 2 must first send message \textit{``4''}
to \textit{ep0} before it can send message \textit{``Go''} to \textit{ep1};
consequently, task 1 is then able to send message \textit{``1''} to
\textit{ep0}. At the end of execution the assertion on line \texttt{12}
of task 0 holds and no error is found. Such intuition is a valid
program execution, and the sequential trace (ignoring endpoint creation)
is given in the first column for each
task in \figref{fig:mcapi} starting on line \texttt{04} of task 2. This single sequential execution is not sufficient to prove correctness
however because of concurrency as other valid traces may exist. How
to test those other traces is the question answered in this paper.

A single execution through a concurrent program can be understood as a
concurrent trace program (CTP) which is a trace divided into its
concurrent components including the requisite control flow on which
the trace is valid \cite{wang:fse09}. The CTP for the trace indicated
in the first column of each task in \figref{fig:mcapi} is shown in
the bottom of the same figure. The CTP is more abstract than the
original C code, and line numbers are now grouped into a program location
(where the number in front of ``\_'' is the thread number and the number behind is the command line)
in each command, but it retains sufficient details to reason about
other possible executions. The assume statement on program location \texttt{0\_5}
of task 0 preserves the control flow of the original trace. In other
words, any alternative trace on the CTP is only valid if the assume
statement holds; otherwise, such an execution belongs to a different
path through the program. An SMT solver is used to enumerate the
possible valid executions in an effort to find one that violates the
assertion on program location \texttt{0\_5} of the CTP for our example.

The novelty of the SMT encoding in this paper is its use of match
pairs rather than the state-based or order-based encoding of prior
work. A match pair is the coupling of a receive to a particular send.
\figref{fig:smt}(a) is the set of possible match pairs for the CTP in
the bottom of \figref{fig:mcapi} for our running example. The set
admits, for example, that \texttt{rcvA} can be matched with either
\texttt{snd1} or \texttt{snd2}. The SMT encoding in this paper asks
the SMT solver to resolve the match pairs for the system.

Prior SMT models of MCAPI program executions implicitly compute
match-pairs by adding possible happens-before relations on sends and
receives with the conditions under which those orderings are
valid. The SMT solver is then asked to resolve the happens-before
relation by choosing specific orders of sends and receives. Match-pair
encoding, as presented in this work, though it has the same
computational complexity as order based encoding (you still need to
figure out match pairs on endpoints), is simpler to reason about
directly rather than implicitly through orders, results in
significantly fewer terms in the SMT problem, and does not restrict
out possible matches in non-blocking actions--the order based encoding
misses valid traces.

\figref{fig:smt}(b) is the stylized SMT encoding for the CTP from our running
example. Not shown are the definitions for \texttt{rcvA}, \texttt{rcvB}, etc.
that are not novel to our solution but are records that group the various
parameters for the respective send and receive calls with an event term to
encode order. In our simple example, the send definitions have the match-pair and event
term unconstrained. Receive definitions leave the event term, match-pair, and final
value of the receive variable unconstrained. All of these unconstrained variables must
be resolved by the SMT solver. The variables ending in ``\_loc'' represent the
event variables for the corresponding MCAPI operations. The \texttt{HB} terms
encodes constraints on the event variables to enforce program order as it
appears in each task of the CTP. For example, by program order, \texttt{rcvA}
happens before the wait command \texttt{wait\_rcvA} for \texttt{rcvA} as shown in \figref{fig:smt}(b). Similarly,
\texttt{wait\_rcvA} happens before \texttt{rcvB}. Any assignment to event
variables must comply with program order constraints.

The \texttt{MATCH} terms in \figref{fig:smt}(b) add constraints
requisite with the specified match pair: the send and
receive endpoints must be the same, the send value must appear in the
receive variable, the send and receive events must be ordered
appropriately, and the two events must be matched together. For
example, matching \texttt{rcvA} with \texttt{snd3} forces the variable
$a$ to have the value 1. The \texttt{MATCH} function encodes this
relationship. The \texttt{ne} term prevents the SMT solver from
finding solutions that use conflicting match pairs. In our example,
the \texttt{ne} term precludes \texttt{rcvA} and \texttt{rcvB} from
matching to the same send. The assume prevents the SMT solver from
finding solutions that are not consistent with control flow, so any
solution that has $b \leq 0$ is infeasible. Finally, the assert is negated
as the goal is to find schedules that violate the property. In our
example, the interest is in solutions the resolve in $a \neq 4$.

The SMT encoding in \figref{fig:smt}(b) can be given to any preferred SMT
solver and a satisfying assignment is a feasible execution that violates an
assertion. For the example in \figref{fig:mcapi} such an execution does exist.
The MCAPI natural language description states that the send operation
\emph{``returns once the buffer can be reused by the application."}. As such,
the return of the send only implies a copy-out of the message buffer
and not a delivery to the intended endpoint; thus, an additional
program execution places the value $1$ in variable $a$ and the value
$4$ in variable $b$. In order to place the value of some send to a variable of a specific receive call, we apply the queue movement in the sequential execution, where the pending sends in the source task are moved to the send queue in the destination task. The first column of each task in the bottom
of the figure gives the sequential program order of each statement for the error trace, and the
second column indicates the appropriate queue movements where the match pairs are implied. For example,
step \texttt{09} of the trace (found in task 0) indicates
that the first item of the pending sends in task 1 is moved to the send queue in task 0, and the wait command on \texttt{rcvA} in program location \texttt{0\_1} matches with \texttt{snd3} (the first item in the send queue in task 0).  Conversations with the
MCAPI designers confirm the intended behavior of the API to include
both program executions of the scenario. To date, there have been
three published verification and analysis tools purpose-built for
MCAPI that omit this less intuitive program
execution~\cite{sharma:fmcad09,elwakil:padtad10,elwakil:atva10}.

Finally, since the computation of match-pairs is expensive to do
precisely, our solution includes the ability to overapproximate the
match-pair set, and then validate satisfying assignments from the SMT
solver to see if they use only feasible match pairs. Such validation
is accomplished by creating a model of the MCAPI library with full
operational semantics. The model is able to detect bad match pairs in
a given program trace. The bad match-pairs can be omitted from the SMT
encoding and the SMT solver re-run to see if another solution exists.
The framework provides a CEGAR loop for MCAPI trace verification. The
remainder of this paper details our MCAPI trace model, encoding, and
an algorithm to compute a precise match pair set.

\begin{figure*}
\begin{center}
\setlength{\tabcolsep}{2pt}
\begin{tabular}[t]{c|c|c}
Task 0 & Task 1 & Task 2 \\
\hline
\scalebox{0.5}{\usebox{\boxTZero}} &
\scalebox{0.5}{\usebox{\boxTOne}}  &
\scalebox{0.5}{\usebox{\boxTTwo}}  \\
& & \\
\hline
\scalebox{0.5}{\usebox{\boxATZero}} &
\scalebox{0.5}{\usebox{\boxATOne}}  &
\scalebox{0.5}{\usebox{\boxATTwo}}
\end{tabular}
\end{center}
\caption{Two versions and traces of the same concurrent system with the top using MCAPI and the bottom using the trace language abstraction derived from the execution order in the first column of the program in the top.}
\label{fig:mcapi}
\end{figure*}

\begin{figure}
\begin{center}
\setlength{\tabcolsep}{25pt}
\begin{tabular}[c]{cc}
\scalebox{0.7}{\usebox{\boxMP}} &
\scalebox{0.7}{\usebox{\boxSMT}} \\ \\
(a) & (b)
\end{tabular}
\end{center}
\caption{A match-pair set and SMT encoding of the system in bottom half of \figref{fig:mcapi}. (a) The match pairs based on end-points. (b) The SMT encoding where \texttt{HB} creates a \emph{happens-before} constraint, \emph{MATCH} creates a match-pair constraint, and \emph{ne} creates a constraint to preclude the use of conflicting match pairs.}
\label{fig:smt}
\end{figure}
