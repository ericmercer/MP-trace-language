\section{Example}

It is a challenge to explain intended behavior in simple scenarios
consisting of a handful of calls when dealing with concurrency. Consider
the MCAPI program execution in \figref{fig:mcapi} that includes three
tasks that use send (\texttt{mcapi\_msg\_sen\\d\_i}) and receive
(\texttt{mcapi\_msg\_recv\_i}) calls to communicate with each other.
Line numbers appear in the first column for each task with the first digit being the task ID, and the
declarations of the local variables are omitted for space. Picking up
the scenario just after the endpoints are defined, lines \texttt{02}
and \texttt{05} receive two messages on the endpoint
\textit{e0} in variables $A$ and $B$ which are
converted to integer values and stored in variables $a$ and
$b$ on lines \texttt{04} and \texttt{07}; task 1 receives one
message on endpoint \textit{e1} in variable \textit{C} on line
\texttt{13} and then sends the message \textit{``1''} on line \texttt{15} to
\textit{e0}; and finally, task 2 sends messages \textit{``4''} and \textit{``Go''} on
lines \texttt{24} and \texttt{26} to endpoints \textit{e0} and
\textit{e1} respectively. The additional code (lines \texttt{08} -
\texttt{09}) asserts properties of the values in $a$ and
$b$. The \texttt{mcapi\_wait} calls block until the associated
send or receive buffer is able to be used. Given the scenario, a developer
might ask the question: \emph{``What are the possible values of
\texttt{a} and \texttt{b} after the scenario completes?''}

\begin{figure}[h]
\begin{center}
\setlength{\tabcolsep}{2pt}
\scriptsize \begin{tabular}[t]{l}
24 $\mathtt{S_{2,4}(0,\&h5)}$ \\
25 $\mathtt{W(\&h5)}$\\
\hline
02 $\mathtt{R_{0,2}(2,\&h1)}$ \\
03 $\mathtt{W(\&h1)}$ \\
\hline
26 $\mathtt{S_{2,6}(1,\&h6)}$ \\
27 $\mathtt{W(\&h6)}$ \\
\hline
04 $\mathtt{a = atoi(A);}$ \\
\hline
13 $\mathtt{R_{1,3}(2,\&h3)}$ \\
14 $\mathtt{W(\&h3)}$ \\
15 $\mathtt{S_{1,5}(0,\&h4)}$ \\
16 $\mathtt{W(\&h4)}$ \\
\hline
05 $\mathtt{R_{0,5}(1,\&h2)}$ \\
06 $\mathtt{W(\&h2)}$ \\
07 $\mathtt{b = atoi(B);}$ \\
08 $\mathtt{assume (b > 0);}$ \\
09 $\mathtt{assert(a == 4);}$ \\
\end{tabular}
\end{center}
\caption{A feasible  execution traces of the MCAPI program execution in \figref{fig:mcapi}}
\label{fig:trace1}
\end{figure}

The intuitive trace is shown in \figref{fig:trace1} using a shorthand
notation for the MCAPI commands: send (denoted as $\mathtt{S}$),
receive (denoted as $\mathtt{R}$), or wait (denoted as $\mathtt{W}$).
The shorthand notation further preserves the thread ID and line number
as follows: for each command $\mathtt{O_{i,j}(k,\&h)}$, $\mathtt{O \in
  \{S,R\}}$ or $\mathtt{W(\&h)}$, $\mathtt{i}$ represents the task ID,
$\mathtt{j}$ represents the source line number, $\mathtt{k}$
represents the destination endpoint, and $\mathtt{h}$ represents the
command handler. A specific destination task ID is indicated in the
notation when a trace is fully resolved, otherwise ``*" notation
indicates that a receive has yet to be matched to a specific send. The
lines in the trace indicate the context switch where a new task
executes.

From the trace, variable \textit{a} should contain $4$ and variable
\textit{b} should contain $1$ since task 2 must first send message
\textit{``4''} to \textit{e0} before it can send message
\textit{``Go''} to \textit{e1}; consequently, task 1 is then able to
send message \textit{``1''} to \textit{e0}. The assume notation
asserts the control flow taken by the program execution. In this
example, the program takes the true branch of the condition on line
\texttt{08}.  At the end of execution the assertion on line
\texttt{09} holds and no error is found.

For the example in \figref{fig:mcapi}, if a zero-buffer
semantics is applied, then the send call on line \texttt{15} cannot match
with the receive call on line \texttt{02}. This blocked match is because
the send call on line \texttt{24} must be completed before
the send call on line \texttt{26} can match with the receive
call on line \texttt{13}.

\begin{figure}[h]
\begin{center}
\setlength{\tabcolsep}{2pt}
\scriptsize \begin{tabular}[t]{l}
24 $\mathtt{S_{2,4}(0,\&h5)}$ \\
25 $\mathtt{W(\&h5)}$ \\
26 $\mathtt{S_{2,6}(1,\&h6)}$ \\
27 $\mathtt{W(\&h6)}$ \\
\hline
13 $\mathtt{R_{1,3}(2,\&h3)}$ \\
14 $\mathtt{W(\&h3)}$ \\
15 $\mathtt{S_{1,5}(0,\&h4)}$ \\
16 $\mathtt{W(\&h4)}$ \\
\hline
02 $\mathtt{R_{0,2}(1,\&h1)}$ \\
03 $\mathtt{W(\&h1)}$ \\
04 $\mathtt{a = atoi(A);}$ \\
05 $\mathtt{R_{0,5}(2,\&h2)}$\\
06 $\mathtt{W(\&h2)}$ \\
07 $\mathtt{b = atoi(B);}$ \\
08 $\mathtt{assume (b > 0);}$ \\
09 $\mathtt{assert(a == 4);}$ \\
\hline
\end{tabular}
\end{center}
\caption{A second feasible  execution traces of the MCAPI program in \figref{fig:mcapi}}
\label{fig:trace2}
\end{figure}

There is another feasible trace for the example shown in
\figref{fig:trace2} which is reachable under the infinite-buffer
semantics. In this trace, the variable \textit{a} contains $1$ instead
of $4$, since the message \textit{``1''} is sent to \textit{e0} after
sending the message \textit{``Go''} to \textit{e1} as it is possible
for the send on line \texttt{24} to be buffered in transit. The MCAPI
specification indicates that the wait on line \texttt{25} returns once
the buffer is available. That only means the message is somewhere in
the MCAPI runtime under the infinite-buffer semantics; it does not
mean the message is delivered. As such, it is possible for the message
to be buffered in transit allowing the send from on line \texttt{15}
to arrive at \textit{e0} first and be received in variable
\textit{``a''}. Such a scenario is a program execution that results in
an assertion failure at line \texttt{09}.

From the discussion above, it is important to consider non-determinism
in the MCAPI runtime when testing or debugging an MCAPI program
execution.  The next section presents an encoding algorithm that takes
an MCAPI program execution with a set of possible send-receive match
pairs and generates an SMT problem that if satisfied proves that
non-determinism can be resolved in a way that violates a user provided
assertion and if unsatisfiable proves the trace correct (meaning the
user assertions hold on the given execution under all possible runtime
behaviors). The encoding can be solved by an SMT solver such as Yices
\cite{dutertre:CAV06} or Z3 \cite{demoura:tacas08}.

%The novelty of the SMT encoding in this paper is its use of match pairs rather than the state-based or order-based encoding of prior work~\cite{elwakil:padtad10,elwakil:atva10}. A match pair is the coupling of a receive to a particular send. \figref{fig:smt}(a) is the set of possible match pairs for the CTP in the bottom of \figref{fig:mcapi} for our running example. The set admits, for example, that \texttt{rcvA} can be matched with either \texttt{snd1} or \texttt{snd2}. The SMT encoding in this paper asks the SMT solver to resolve the match pairs for the system.


