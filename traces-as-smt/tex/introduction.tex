\section{Introduction}
Embedded devices fill all sorts of crucial roles in our lives. They exist as
medical devices, as network infrastructure, and they control our automobiles.
Embedded devices continue to become more powerful as computing hardware becomes
smaller and more modular. It is now becoming commonplace to find multiple
processing units inside a single device. The Multicore Association (MCA) is an
industry group that has formed to define specifications for low-level
communication, resource management, and task management for embedded heterogeneous
multicore devices \cite{mca}.

One specification that the MCA has released is the Multicore
Association Communications API (MCAPI) \cite{mcapi}. The specification
defines types and functions for simple message passing operations between different
computing entities within a device. Messages can be passed across
persistent channels that force an ordering of the messages, or they
can be passed to specific \emph{endpoints} within the system. The
specification places few ordering constraints on messages passed from
one endpoint to another. This freedom introduces the possibility of a race between multiple messages to common endpoints thus giving rise to non-deterministic behavior in the runtime\ \cite{netzer:spdt96}. If an
application has non-determinism, it is not possible
to test and debug such an application without a way to directly (or
indirectly) control the MCAPI runtime.

There are two ways to implement the MCAPI semantics: infinite-buffer
semantics (the message is copied into a runtime buffer on the API
call) and zero-buffer semantics (the message has no buffering)
\cite{sarvani:fm09}.  An infinite-buffer semantics provides more
non-deterministic behaviors in matching send and receives because the
runtime can arbitrarily delay a send to create interesting (and
unexpected) send reorderings. The zero-buffer semantics follow
intuitive message orderings as a send and receive essentially
rendezvous.

Sharma et al. propose a method to indirectly control the MCAPI runtime
to verify MCAPI programs under zero-buffer semantics
\cite{sharma:fmcad09}. As the work does not address infinite-buffer
semantics, it is somewhat limited in its application. The work does
provide a dynamic partial order reduction for the model checker, but
such a reduction is not sufficient to control state space explosion in
the presence of even moderate non-determinism between message sends
and receives. A key insight from the approach is its direct use of
match pairs--couplings for potential sends and receives.

Wang \textit{et al.} propose an alternative method for resolving
non-determinism for program verification using symbolic methods in the context
of shared memory systems \cite{wang:fse09}. The work observes a
program trace, builds a partial order from that trace called a
concurrent trace program (CTP), and then creates an SMT problem from
the CTP that if satisfied indicates a property violation. 

\examplefigone

Elwakil \textit{et al.} extend the work of Wang \textit{et al.} to
message passing and claim the encoding supports both infinite and zero buffer semantics. A
careful analysis of the encoding, however, shows it to not work under
infinite-buffer semantics and to miss behaviors under zero-buffer
semantics \cite{elwakil:padtad10}. Interestingly, the encoding assumes
the user provides a precise set of match pairs as input with the
program trace, and it then uses those match pairs in a non-obvious way
to constrain the happens-before relation in the encoding. The work
does not discuss how to generate the match pairs, which is a
non-trivial input to manually generate for large or complex program
traces. An early proof claims that the problem of finding a precise
set of match pairs given a program trace is NP-complete
\cite{match-pair-np-complete}.

 This paper presents a proof that resolving non-determinism in message
 passing programs in a way that meets all assertions is
 NP-complete. The paper then presents an SMT encoding for MCAPI
 program executions that works for both zero and infinite buffer
 semantics. The encoding does require an input set of match pairs as
 in prior work, but unlike prior work, the match-set can be
 over-approximated and the encoding is still sound and complete. The
 encoding requires fewer terms to capture all possible program
 behavior when compared to other proposed methods making it more
 efficient in the SMT solver.  To address the problem of generating
 match pairs, an algorithm to generate the over-approximated set is
 given. To summarize, the main contributions in this paper are

\begin{enumerate}
%\item An MCAPI program that introduces non-deterministic behavior.
%\item A trace language framework that simulates and verifies an execution trace of an MCAPI program.
%\item A modified trace language framework that generates an SMT problem for a single trace in the machine executions.
\item a proof that the problem of matching sends to receives in a way that meets assertions is NP-complete;
\item a correct and efficient SMT encoding of an MCAPI program
  execution that detects all program errors under zero or infinite
  buffer semantics given the input set of potential match pairs
  contains at least the precise set of match pairs; and
\item an $O(N^2)$ algorithm to generate an over-approximation of
  possible match pairs, where $N$ is the size of the execution trace
  in lines of code.
\end{enumerate}

\noindent \textbf{Organization}: Section 2 gives an example. Section 3
shows the NP-completeness reduction. Section 4 gives the
encoding. Section 5 shows how to generate match-pairs.  Section 6
presents the results. Section 7 discusses related work. And Section 8
is conclusions and future work.

