\section{Introduction}
Embedded devices fill all sorts of crucial roles in our lives. They exist as
medical devices, as network infrastructure, and they control our automobiles.
Embedded devices continue to become more powerful as computing hardware becomes
smaller and more modular. It is now becoming commonplace to find multiple
processing units inside a single device. The Multicore Association (MCA) is an
industry group that has formed to define specifications for low-level
communication, resource management, and task management for embedded heterogeneous
multicore devices \cite{mca}.

One specification that the MCA has released is the Multicore
Association Communications API (MCAPI) \cite{mcapi}. The specification
defines types and functions for simple message passing operations between different
computing entities within a device. Messages can be passed across
persistent channels that force an ordering of the messages, or they
can be passed to specific \emph{endpoints} within the system. The
specification places few ordering constraints on messages passed from
one endpoint to another. This freedom introduces the possibility of a race between multiple messages to common endpoints thus giving rise to non-deterministic behavior in the runtime\cite{netzer:spdt96}. If an
application has non-determinism, it is not possible
to test and debug such an application without a way to directly (or
indirectly) control the MCAPI runtime.

There are two ways to implement the MCAPI semantics: infinite-buffer
semantics (the message is copied into a runtime buffer on the API
call) and zero-buffer semantics (the message has no buffering)
\cite{sarvani:fm09}.  An infinite-buffer semantics provides more
non-deterministic behaviors in matching send and receives because the
runtime can arbitrarily delay a send to create interesting (and
unexpected) send reorderings. The zero-buffer semantics follow
intuitive message orderings as a send and receive essentially
rendezvous.

Sharma et al. propose a method of indirectly controlling the MCAPI
runtime to verify MCAPI programs under zero-buffer semantics, but it
was later discovered that this method does not completely explore the
entire execution space of certain kinds of MCAPI programs
\cite{sharma:fmcad09}. A key insight of the approach, however, is in
its use of match pairs--couplings for potential sends and receives.
Wang \textit{et al.} propose an alternative method for resolving
non-determinism for program verification using symbolic methods in the context
of shared memory systems \cite{wang:fse09}. The work observes a
program trace, builds a partial order from that trace called a
concurrent trace program (CTP), and then creates an SMT problem from
the CTP that if satisfied indicates a property violation. Elwakil \textit{et
al.} extend the work of Wang \textit{et al.} to message passing in zero-buffer semantics, but the
proposed encoding does not capture all allowed MCAPI program
executions \cite{elwakil:padtad10}. Further, the method assumes that
the user is able to provide the exact set of match pairs. Such an
assumption is not reasonable for large complex program traces.

 This paper presents an SMT encoding for MCAPI program executions that
 works for both zero and infinite buffer semantics. Further, the
 encoding requires fewer terms to capture all possible program
 behavior when compared to other proposed methods.  As a result, the
 SMT encoding in this paper is both correct in that it enumerates all
 possible executions allowed by the MCAPI specification in either zero
 or infinite buffer semantics, and it is efficient since it uses
 dramatically fewer terms in the encoding to reduce the memory and
 running time in the SMT solver. To summarize, the main contributions
 in this paper are

\examplefig1

\begin{enumerate}
%\item An MCAPI program that introduces non-deterministic behavior.
%\item A trace language framework that simulates and verifies an execution trace of an MCAPI program.
%\item A modified trace language framework that generates an SMT problem for a single trace in the machine executions.
\item a correct and efficient SMT encoding of an MCAPI program
  execution that detects all program errors if the provided match
  pairs are precise or over approximated under zero or infinite buffer
  semantics; and
\item an $O(N^2)$ algorithm to generate an over-approximation of
  possible match-pairs, where $N$ is the size of the execution trace
  in lines of code.
\end{enumerate}

The rest of the paper is organized as follows: Section 2 presents an
MCAPI program execution in which two program traces exist due to
non-determinism in the MCAPI runtime. Section 3 defines an SMT
encoding of an MCAPI program execution that captures non-determinism
in message order by defining the set of possible match pairs between
sends and receives on common endpoints. It is proved that an
over-approximated match set in our SMT encoding of an MCAPI program
execution can reflect the actual execution traces as the true set
does. Section 4 provides a solution to the outstanding problem in
other encodings of generating feasible match pairs to use in the
encoding. It presents a $O(N^2)$ algorithm that over-approximates the
precise match set, where $N$ is the size of the execution trace in
lines of code. Section 5 presents the experimental results that show
our encoding to be correct and efficient under both zero-buffer and
infinite-buffer semantics, as other existing encodings omit critical
program behaviors, require more SMT clauses and only support
zero-buffer semantics. Section 6 discusses related work, and Section 7
presents our conclusions and future work.

