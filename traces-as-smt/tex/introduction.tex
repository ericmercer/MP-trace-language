\section{Introduction}
Embedded devices fill all sorts of crucial roles in our lives. They exist as
medical devices, as network infrastructure, and they control our automobiles.
Embedded devices continue to become more powerful as computing hardware becomes
smaller and more modular. It is now becoming commonplace to find multiple
processing units inside a single device. The Multicore Association (MCA) \cite{mca} is an
industry group that has formed to define specifications for low-level
communication, resource management, and task management for
multicore devices.

One specification that the MCA has released is the Multicore Association
Communications API (MCAPI) \cite{mcapi}. The specification defines basic data type, data
structures, and functions that can be used to perform simple message passing
operations between different computing entities within a device. Messages can be
passed across persistent channels that force an ordering of the messages, or
they can be passed to specific \emph{endpoints} within the system. The specification places few ordering constraints on messages passed from one
endpoint to another. This freedom introduces into the system possibilities that two or more messages are racing if the order of their arrival at the destinations are non-deterministic \cite{netzer:spdt96}. From our observation, few MCAPI applications have this non-determinism in practice. However, if it is implemented, it is not possible to test and debug the program executions without a way to explore this non-determinism in the MCAPI runtime.

Sharma et al. created a method of using concrete execution to verify MCAPI
programs, but it was later discovered that this method does not completely
explore the entire execution space of certain kinds of MCAPI programs \cite{sharma:fmcad09}. The method provides match pairs -- couplings for potential sends and receives that we use in our work. Instead of exploring all relevant interleavings of a program in the concrete execution in \cite{sharma:fmcad09}, Wang et al. provided a symbolic algorithm that verifies each partitioned concurrent trace program (CTP) with shared memory semantics using a satisfiability solver \cite{wang:fse09}. Elwakil et al. also defined a method of representing MCAPI program executions as Satisfiability Modulo Theory (SMT) problems building on the method of \cite{wang:fse09} adapting it to message passing, but this method does not capture all allowed MCAPI program executions \cite{elwakil:padtad10}. Further, this method assumes that the user is able to provide the exact set of match pairs. Such an assumption is not reasonable for large complex program traces. In this paper, however, we provide an algorithm that generates an over-approximated set of match pairs. In order to prevent the problems discussed above, this paper discusses a new method that provides both efficiency and correctness.

There are two ways to implement the MCAPI semantics for matching sends to receives including an \textit{infinite-buffer} semantics (the message is copied into a runtime provided buffer on the API call) or a \textit{zero-buffer} semantics (the message is copied into a endpoint provided buffer) \cite{sarvani:fm09}.  An \textit{infinite-buffer} semantics provides more non-deterministic behaviors because a particular endpoint can receive more messages under the circumstance of \textit{infinite-buffer} semantics while the \textit{zero-buffer} semantics can only allow one message to be received by a specific endpoint at a time. We present an encoding for MCAPI program executions that works for both \textit{zero-buffer} and \textit{infinite-buffer} semantics that requires fewer terms than \cite{elwakil:padtad10} but includes all possible program executions unlike \cite{elwakil:padtad10}.  In other words, the solution for our encoding provides efficiency. Our encoding uses match pairs from \cite{sharma:fmcad09} that captures the non-determinism of program executions. Most importantly, this method correctly captures all possible execution traces allowed by the MCAPI specification in a trace of a concurrent program execution as an SMT problem enabling the exploration of inherent non-determinism that may exist in any MCAPI runtime implementation.
Our main contributions in this paper include:

\begin{enumerate}
%\item An MCAPI program that introduces non-deterministic behavior.
%\item A trace language framework that simulates and verifies an execution trace of an MCAPI program.
%\item A modified trace language framework that generates an SMT problem for a single trace in the machine executions.
\item  A correct and efficient SMT encoding of an MCAPI program execution that detects all program errors if the user provided match pairs are precise or over approximated.
\item  An $O(N^2)$ algorithm to generate an over-approximation of possible match-pairs, where $N$ is the size of the execution trace as lines of code.
\end{enumerate}

The rest of the paper is organized as follows:
Section 2 presents an MCAPI program execution in which two program traces exist due to non-determinism in the MCAPI runtime. Section 3 defines an SMT encoding of an MCAPI program execution that captures non-determinism in message order by defining the set of possible match pairs between sends and receives on a common endpoint. It is proved that an over-approximated match set in our SMT encoding of an MCAPI program execution can reflect the actual execution traces as the true set does. Section 4 solves the outstanding problem in other encodings of generating feasible match pairs to use in the encoding. It presents a $O(N^2)$ algorithm that over-approximates the precise match set, where $N$ is the size of the execution trace in lines of code. Section 5 presents the experimental results that show our encoding to be correct and efficient, as other existing encodings both omit critical program behaviors and require more SMT clauses. Section 6 discusses related
work, and Section 7 presents our conclusions and future work.


