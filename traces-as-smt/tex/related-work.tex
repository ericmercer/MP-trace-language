\section{Related Work}
Morse \emph{et al.} provided a formal modeling paradigm that is callable from
the C language for the MCAPI interface \cite{morse:vmcai12}. This
model correctly captures the behavior of the interface and can be
applied to model checking C programs that use the API. The work is a
direct application of model checking and directly enumerates the
non-determinism in the runtime to construct an exhaustive proof. The
SMT encoding in this paper pushes that complexity to the SMT solver
and leverages recent advances in SMT technology to find a satisfying
assignment.

Vakkalanka \emph{et al.} define POE, a dynamic partial order algorithm for
MPI programs \cite{sarvani:fm09, flanagan:popl05}. As MPI is similar
to MCAPI, though more expressive, the algorithm is relevant to MCAPI
verification \cite{mpi}. Intuitively, the POE algorithm, as a model
checker, enumerates non-determinism in the MPI specification under
zero-buffer semantics with the goal to detect deadlock. The algorithm
is not suitable to infinite-buffer semantics as it misses
behavior. The relatively efficiency, in practice, of the model checker
enumerating non-determinism versus the SMT solver is an open question
for the POE work.

Key to the POE algorithm is the notion of a
\textit{intra-happens-before-order}. The order is a partial order over
MPI events that generalizes to any control flow path through the
program.  The SMT encoding in this builds the happens-before relation
for a given control flow path and does not reason directly over all
control flow paths in the program. A secondary analysis is needed to
find program inputs sufficient to enumerate all reachable control flow
paths in order to apply the SMT encoding to complete program
verification.

Sharma \emph{et al.} present an dynamic model checker for MCAPI programs built
on top of the MCA provided MCAPI runtime \cite{sharma:fmcad09}. MCC
systematically enumerates all non-determinism in the MCAPI runtime
under zero-buffer semantics. It employs a novel dynamic partial order
reduction to avoid enumerating redundant message orders. This work
claims SMT technology is more efficient in practice resolving
non-determinism in a away to violate correctness properties.

Wang \emph{et al.} present an SMT encoding for shared memory semantics for a
given input trace from a multi-threaded program \cite{wang:fse09}. As
mentioned previously, the program is partitioned into several
concurrent trace programs, and the encoding for each program is
verified using SMT technology. Elwakil \emph{et al.} extend the encoding to
message passing programs using the MCAPI semantics
\cite{elwakil:atva10,elwakil:padtad10}. The comparison to the encoding
in this work is already discussed previously.

There is a rich body of literature for SMT/SAT based Bounded Model
Checking. Burckhardt \emph{et al.} exhaustively check all executions of a
test program by translating the program implementation into SAT
formulas \cite{burckhardt:pldi07}. The approach relies on
counter-examples from the solvers the refine the encoding. The SMT
encoding in this work is able to directly resolve the match-pair set
over-approximation directly without needing to check a
counter-example.

Dubrovin \emph{et al.} give a method to translate an asynchronous system into
a transition formula over three partial order semantics
\cite{heljanko:scp}. The encoding adds constraints to compress the
search space and decrease the bound on the program unwinding. The
encoding in this paper operates on a program execution and does not
need to resolve a bound.

Kahlon \emph{et al.} presented a partial order reduction, \textit{MPOR}, that
operates in the bounded model checking space \cite{kahlon:cav09}.  It
guarantees that exactly one execution is calculated per each
Mazurkiewicz trace to reduce the search space. It would be interesting
to see if MPOR is able to extend to message passing semantics. Other
work in bounded model checking explores heap-manipulating programs and
challenges in sequential systems code \cite{lahiri:popl08, lahiri:cav11}.

The application of static analysis is another interesting thread of
research to test or debug message passing programs with some work in
the MPI domain \cite{zhang:ppopp07, greg:cgo09, gray:lctes11}. The
work is important as it lays the foundation for refining match-pair
sets to only include those that cannot be statically pruned.
