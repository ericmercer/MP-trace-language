\section{Related Work}
Morse et al. provided a formal modeling paradigm that is callable from C language for the MCAPI interface \cite{morse:vmcai12}. This model correctly captures the behavior of the interface and can be applied for model checking analysis for C programs that use the API.

In \cite{sarvani:fm09}, Vakkalanka et al. provided POE, a DPOR algorithm \cite{flanagan:popl05} of their dynamic verifier, ISP, for MPI programs\cite{mpi}. another message passing interface standard. POE explores all interleavings of an MPI program, and is able to detect hidden deadlocks under zero-buffer setting. If infinite-buffer is applied, however, some interleavings may not be found. \cite{sarvani:fm09} also defines \textit{Intra-Happens-Before-Order}(\textit{Intra-HB}), a set of partial orders of MPI events that constrains the Happens-Before relations between commands in any control flow path. Those relations are essential to their POE algorithm of their dynamic verifier ISP. Our SMT structure encodes all the relations in \textit{Intra-HB}, such that it can follow any control flow path executed by the algorithm in \cite{sarvani:fm09}.

In \cite{sharma:fmcad09}, Sharma et al. provided MCC, a dynamic verifier for MCAPI programs. MCC systematically explores all interleavings of an MCAPI program by concretely executing the program repeatedly. MCC uses DPOR \cite{flanagan:popl05} to reduce the redundant interleavings of the execution, however, it does not include all possible traces allowed by the MCAPI specification. Wang et al. provided a symbolic algorithm that detects errors in all feasible permutations of statements in an execution trace in the shared memory system \cite{wang:fse09}. In this method, the program is partitioned into several concurrent trace programs (CTPs), and the encoding for each CTP is verified using a satisfiability solver. Elwakil et al. provided a similar work with ours \cite{elwakil:atva10,elwakil:padtad10}. In their work, the method of \cite{wang:fse09} is used and adapted to the message passing system. As shown in the previous section of this paper, their method does not correctly encode all possible execution traces of an MCAPI program.

The SMT/SAT based Bounded Model Checking is one avenue of verifying and debugging systems. Burckhardt et al. presented CheckFence prototype in \cite{burckhardt:pldi07}, which exhaustively checks all executions of a test program by translating the program implementation into SAT formulas. The standard SAT solver can construct counterexample if incorrect executions exist. Instead of over-approximating at the beginning and then further compressing the observations, CheckFence in \cite{burckhardt:pldi07} increments the observations each time step by adding constraints to SAT formulas. Dubrovin et al. provides a method in \cite{heljanko:scp} that translates an asynchronous system into a transition formula for three partial order semantics. Other than adding constraints to the SAT/SMT formulas in order to compress the search space, the method in \cite{heljanko:scp} decreases the search bound by allowing several actions to be executed simultaneously within one step. Kahlon et al. presented a Partial Order Reduction method called \textit{MPOR} in \cite{kahlon:cav09}. This method cannot only be applied to the explicit state space search as other partial order reduction methods do, but also can be applied to the SMT/SAT based Bounded Model Checking. \textit{MPOR} guarantees that exactly one execution is calculated per each Mazurkiewicz trace, in order to reduce the search space. There are several applications of the SMT/SAT based Bounded Model Checking. \cite{lahiri:popl08} presented a precise verification of heap-manipulating programs using SMT solvers. \cite{lahiri:cav11} presented some challenges in SMT-based verification for sequential system code, and tackled these issues by extending the standard SMT solvers.

The application of static analysis is another interesting thread of research to test or debug message passing programs. \cite{zhang:ppopp07} and \cite{greg:cgo09} are the approaches for MPI. \cite{gray:lctes11} presented a system that uses static analysis to determine offline the topology of the communications and nodes in the input MCAPI program.
