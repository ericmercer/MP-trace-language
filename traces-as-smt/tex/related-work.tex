\section{Related Work}
Morse et al. provided a formal modeling paradigm that is callable from C language for the MCAPI interface \cite{morse:vmcai12}. This model correctly captures the behavior of the interface and can be applied for model checking analysis for C programs that use the API.

In \cite{sharma:fmcad09}, Sharma et al. provided MCC, a dynamic verifier for MCAPI programs. MCC systematically explores all interleavings of an MCAPI program by concretely executing the program repeatedly. MCC also uses DPOR \cite{flanagan:popl05} to reduce the redundant interleavings of the execution. Wang et al. provided a symbolic algorithm that detects errors in all feasible permutations of statements in an execution trace in the shared memory system \cite{wang:fse09}. In this method, the program is partitioned into several concurrent trace programs (CTPs), and the encoding for each CTP is verified using a satisfiability solver. Elwakil et al. provided a similar work with ours \cite{elwakil:atva10,elwakil:padtad10}. In their work, the method of \cite{wang:fse09} is used and adapted to the message passing system. As shown in the previous section of this paper, their method, however, does not correctly encode all possible execution traces of an MCAPI program. 

The application of static analysis is another interesting avenue of research to test or debug message passing programs. \cite{zhang:ppopp07} and \cite{greg:cgo09} are the approaches for MPI \cite{mpi}, another message passing interface standard. \cite{gray:lctes11} presented a system that uses static analysis to determine offline the topology of the communications and nodes in the input MCAPI program.
