\newcommand{\epsnd}{\textit{Pnd\_s}}
\newcommand{\eprcv}{\textit{Pnd\_r}}
\newcommand{\qset}{\textit{Q\_s}}
\newcommand{\snd}{\textit{snd}}
\newcommand{\rcv}{\textit{rcv}}
\newcommand{\q}{\textit{q}}
\newcommand{\frm}{\textit{frm}}
\newcommand{\vbot}{\textit{v$-\bot$}}
\newcommand{\status}{\textit{s}}
\newcommand{\last}{\textit{l}}
\newcommand{\error}{\ensuremath{\mathbf{error}}}
\newcommand{\aidmap}{\textit{A}}
\newcommand{\thread}{\ensuremath{\mathit{t}}}
\newcommand{\aid}{\ensuremath{\mathit{a}}}
\newcommand{\ploc}{\ensuremath{\rho}}
\newcommand{\loc}{\ensuremath{\mathit{l}}}
\newcommand{\cmd}{\ensuremath{\mathit{c}}}
\newcommand{\op}{\ensuremath{\mathit{op}}}
\newcommand{\applyop}{\ensuremath{\mathrm{op}}}
\newcommand{\wait}{\ensuremath{\mathbf{wait}}}
\newcommand{\sendi}{\ensuremath{\mathbf{sndi}}}
\newcommand{\recvi}{\ensuremath{\mathbf{rcvi}}}
\newcommand{\traceentry}{\ensuremath{\sigma}}
\newcommand{\movelist}{\ensuremath{\delta}}
\newcommand{\assume}{\ensuremath{\mathit{assume}}}
\newcommand{\checkassume}{\ensuremath{\mathrm{assume}}}
\newcommand{\assert}{\ensuremath{\mathit{assert}}}
\newcommand{\checkassert}{\ensuremath{\mathrm{assert}}}
\newcommand{\ep}{\ensuremath{\mathbf{\gamma}}}
\newcommand{\src}{\ensuremath{\alpha}}
\newcommand{\dst}{\ensuremath{\beta}}
\newcommand{\true}{\ensuremath{\mathbf{true}}}
\newcommand{\false}{\ensuremath{\mathbf{false}}}
\newcommand{\match}{\ensuremath{\mathrm{m}}}
\newcommand{\matchl}{\ensuremath{\mathrm{match}}}
\newcommand{\findrecv}{\ensuremath{\mathrm{search_r}}}
\newcommand{\smt}{\ensuremath{\mathit{smt}}}


\newcommand{\addsend}{\ensuremath{\mathrm{addsend}}}
\newcommand{\getsend}{\ensuremath{\mathrm{getsend}}}
\newcommand{\removesend}{\ensuremath{\mathrm{removesend}}}
\newcommand{\addrecv}{\ensuremath{\mathrm{addrecv}}}
\newcommand{\getrecv}{\ensuremath{\mathrm{getrecv}}}
\newcommand{\removerecv}{\ensuremath{\mathrm{removefrecv}}}
\newcommand{\hextend}{\ensuremath{\mathrm{hextend}}}
\newcommand{\hlookup}{\ensuremath{\mathrm{hlookup}}}
\newcommand{\etalookup}{\ensuremath{\mathrm{etalookup}}}
\newcommand{\checkmatch}{\ensuremath{\mathrm{checkmatch}}}
\newcommand{\getep}{\ensuremath{\mathrm{getEP}}}
\newcommand{\getmarkremove}{\ensuremath{\mathrm{getMarkRemove}}}
\newcommand{\getlastsendreplace}{\ensuremath{\mathrm{getlastsend/replace}}}
\newcommand{\statuschange}{\left\{ \begin{array}{ll}  \status &\ \mathrm{if}\ |\epsnd(\dst)(\src)|>0\\
   \mathit{error} &\  \mathrm{otherwise}\end{array}\right .}

\newcommand{\reduce}[1]{\ensuremath{\rightarrow_{#1}}}
% Multiple reductions
\newcommand{\reduceK}[1]{\ensuremath{\rightarrow_{#1}^{*}}}
% Non-deterministic reductions
\newcommand{\reduceN}[1]{\ensuremath{\dashrightarrow_{#1}}}
\newcommand{\reduceNK}[1]{\ensuremath{\dashrightarrow_{#1}^{*}}}
\newcommand{\mt}{\ensuremath{\emptyset}}
\newcommand{\trace}{\ensuremath{\mathit{trace}}}
\newcommand{\movebot}{\ensuremath{\mathit{m}}}
\newcommand{\ret}{\ensuremath{\mathbf{ret}}}

\newcommand{\defs}{\ensuremath{\mathrm{defs}}}
\newcommand{\any}{\ensuremath{\mathrm{any}}}
\newcommand{\negate}{\ensuremath{\mathrm{not}}}
\newcommand{\define}{\ensuremath{\mathrm{define}}}
\newcommand{\andd}{\ensuremath{\mathrm{and}}}
\newcommand{\select}{\ensuremath{\mathrm{select}}}
\newcommand{\HB}{\ensuremath{\mathrm{HB}}}
\newcommand{\getlast}{\ensuremath{\mathrm{last}}}
\newcommand{\MATCH}{\ensuremath{\mathrm{MATCH}}}
\newcommand{\ml}{\ensuremath{\mathit{ML}}}
\newcommand{\matchpair}{\ensuremath{\mathrm{MP}}}
\newcommand{\recv}{\ensuremath{\mathrm{\mu}}}
\newcommand{\send}{\ensuremath{\mathrm{\nu}}}



\newsavebox{\boxSMTa}
\begin{lrbox}{\boxSMTa}
\normalsize
\begin{tabular}[c]{l}
\texttt{\textit{defs} is not shown;}\\
\\
\texttt{\textit{constraints};}\\
\texttt{00 $\mathtt{event_{R_{0,2}}}$ $\mathtt{\prec_\mathtt{HB}}$ $\mathtt{event_{W(\&h1)}}$}\\
\texttt{01 $\mathtt{event_{W(\&h1)}}$.event $\mathtt{\prec_\mathtt{HB}}$ $\mathtt{event_{R_{0,5}}}$}\\
\texttt{02 $\mathtt{event_{R_{0,5}}}$.event $\mathtt{\prec_\mathtt{HB}}$ $\mathtt{event_{W(\&h2)}}$}\\
\texttt{03 $\mathtt{event_{W(\&h2)}}$.event $\mathtt{\prec_\mathtt{HB}}$ $\mathtt{event_{assume}}$}\\
\texttt{04 $\mathtt{event_{assume}}$ $\mathtt{\prec_\mathtt{HB}}$ $\mathtt{event_{assert}}$}\\
\texttt{05 $\mathtt{event_{R_{1,3}}}$ $\mathtt{\prec_\mathtt{HB}}$ $\mathtt{event_{W(\&h3)}}$}\\
\texttt{06 $\mathtt{event_{W(\&h3)}}$ $\mathtt{\prec_\mathtt{HB}}$ $\mathtt{event_{S_{1,5}}}$}\\
\texttt{07 $\mathtt{event_{S_{1,5}}}$ $\mathtt{\prec_\mathtt{HB}}$ $\mathtt{event_{W(\&h4)}}$}\\
\texttt{08 $\mathtt{event_{S_{2,4}}}$ $\mathtt{\prec_\mathtt{HB}}$ $\mathtt{event_{W(\&h5)}}$}\\
\texttt{09 $\mathtt{event_{W(\&h5)}}$ $\mathtt{\prec_\mathtt{HB}}$ $\mathtt{event_{S_{2,7}}}$}\\
\texttt{10 $\mathtt{event_{S_{2,7}}}$ $\mathtt{\prec_\mathtt{HB}}$ $\mathtt{event_{W(\&h6)}}$}\\
\texttt{11 (assert (> b 0))}\\
\texttt{12 (assert (not (= a 4)))}\\
\\
\texttt{\textit{match};}\\
\texttt{13 $\langle\mathtt{R_{0,2}}$,$\mathtt{S_{2,4}}\rangle$}\\
\texttt{14 $\langle\mathtt{R_{0,5}}$,$\mathtt{S_{1,5}}\rangle$}\\
\texttt{15 $\langle\mathtt{R_{1,3}}$,$\mathtt{S_{2,7}}\rangle$}\\

\end{tabular}
\end{lrbox}

\newsavebox{\boxSMTb}
\begin{lrbox}{\boxSMTb}
\normalsize
\begin{tabular}[c]{l}
\texttt{\textit{defs} is not shown;}\\
\\
\texttt{\textit{constraints};}\\
\texttt{00 $\mathtt{event_{R_{0,2}}}$ $\mathtt{\prec_\mathtt{HB}}$ $\mathtt{event_{W(\&h1)}}$}\\
\texttt{01 $\mathtt{event_{W(\&h1)}}$.event $\mathtt{\prec_\mathtt{HB}}$ $\mathtt{event_{R_{0,5}}}$}\\
\texttt{02 $\mathtt{event_{R_{0,5}}}$.event $\mathtt{\prec_\mathtt{HB}}$ $\mathtt{event_{W(\&h2)}}$}\\
\texttt{03 $\mathtt{event_{W(\&h2)}}$.event $\mathtt{\prec_\mathtt{HB}}$ $\mathtt{event_{assume}}$}\\
\texttt{04 $\mathtt{event_{assume}}$ $\mathtt{\prec_\mathtt{HB}}$ $\mathtt{event_{assert}}$}\\
\texttt{05 $\mathtt{event_{R_{1,3}}}$ $\mathtt{\prec_\mathtt{HB}}$ $\mathtt{event_{W(\&h3)}}$}\\
\texttt{06 $\mathtt{event_{W(\&h3)}}$ $\mathtt{\prec_\mathtt{HB}}$ $\mathtt{event_{S_{1,5}}}$}\\
\texttt{07 $\mathtt{event_{S_{1,5}}}$ $\mathtt{\prec_\mathtt{HB}}$ $\mathtt{event_{W(\&h4)}}$}\\
\texttt{08 $\mathtt{event_{S_{2,4}}}$ $\mathtt{\prec_\mathtt{HB}}$ $\mathtt{event_{W(\&h5)}}$}\\
\texttt{09 $\mathtt{event_{W(\&h5)}}$ $\mathtt{\prec_\mathtt{HB}}$ $\mathtt{event_{S_{2,7}}}$}\\
\texttt{10 $\mathtt{event_{S_{2,7}}}$ $\mathtt{\prec_\mathtt{HB}}$ $\mathtt{event_{W(\&h6)}}$}\\
\texttt{11 (assert (> b 0))}\\
\texttt{12 (assert (not (= a 4)))}\\
\\
\texttt{\textit{match};}\\
\texttt{13 $\langle\mathtt{R_{0,2}}$,$\mathtt{S_{1,5}}\rangle$}\\
\texttt{14 $\langle\mathtt{R_{0,5}}$,$\mathtt{S_{2,4}}\rangle$}\\
\texttt{15 $\langle\mathtt{R_{1,3}}$,$\mathtt{S_{2,7}}\rangle$}\\

\end{tabular}
\end{lrbox}

\newsavebox{\boxSMTc}
\begin{lrbox}{\boxSMTc}
\normalsize
\begin{tabular}[t]{l}
\texttt{\textit{defs} is not shown;}\\
\\
\texttt{\textit{constraints};}\\
\texttt{00 $\mathtt{event_{R_{0,2}}}$ $\mathtt{\prec_\mathtt{HB}}$ $\mathtt{event_{W(\&h1)}}$}\\
\texttt{01 $\mathtt{event_{W(\&h1)}}$.event $\mathtt{\prec_\mathtt{HB}}$ $\mathtt{event_{R_{0,5}}}$}\\
\texttt{02 $\mathtt{event_{R_{0,5}}}$.event $\mathtt{\prec_\mathtt{HB}}$ $\mathtt{event_{W(\&h2)}}$}\\
\texttt{03 $\mathtt{event_{W(\&h2)}}$.event $\mathtt{\prec_\mathtt{HB}}$ $\mathtt{event_{assume}}$}\\
\texttt{04 $\mathtt{event_{assume}}$ $\mathtt{\prec_\mathtt{HB}}$ $\mathtt{event_{assert}}$}\\
\texttt{05 $\mathtt{event_{R_{1,3}}}$ $\mathtt{\prec_\mathtt{HB}}$ $\mathtt{event_{W(\&h3)}}$}\\
\texttt{06 $\mathtt{event_{W(\&h3)}}$ $\mathtt{\prec_\mathtt{HB}}$ $\mathtt{event_{S_{1,5}}}$}\\
\texttt{07 $\mathtt{event_{S_{1,5}}}$ $\mathtt{\prec_\mathtt{HB}}$ $\mathtt{event_{W(\&h4)}}$}\\
\texttt{08 $\mathtt{event_{S_{2,4}}}$ $\mathtt{\prec_\mathtt{HB}}$ $\mathtt{event_{W(\&h5)}}$}\\
\texttt{09 $\mathtt{event_{W(\&h5)}}$ $\mathtt{\prec_\mathtt{HB}}$ $\mathtt{event_{S_{2,7}}}$}\\
\texttt{10 $\mathtt{event_{S_{2,7}}}$ $\mathtt{\prec_\mathtt{HB}}$ $\mathtt{event_{W(\&h6)}}$}\\
\texttt{11 (assert (> b 0))}\\
\texttt{12 (assert (not (= a 4)))}\\
\\
\texttt{\textit{match};}\\
\texttt{13} $\langle\mathtt{R_{0,2}}$,$\mathtt{S_{2,4}}\rangle\vee\langle\mathtt{R_{0,2}}$,$\mathtt{S_{1,5}}\rangle$\\
\texttt{14} $\langle\mathtt{R_{0,5}}$,$\mathtt{S_{2,4}}\rangle\vee\langle\mathtt{R_{0,5}}$,$\mathtt{S_{1,5}}\rangle$\\
\texttt{15} $\langle\mathtt{R_{1,3}}$,$\mathtt{S_{2,7}}\rangle$\\

\end{tabular}
\end{lrbox}

\newsavebox{\boxMP}
\begin{lrbox}{\boxMP}
\normalsize
\begin{tabular}[t]{l}
\\\\\\\\\\\\\\\\\\
$\langle\mathtt{R_{0,2}}$, $\mathtt{S_{2,4}}\rangle$\\
$\langle\mathtt{R_{0,2}}$, $\mathtt{S_{1,5}}\rangle$\\
\\
$\langle\mathtt{R_{0,5}}$, $\mathtt{S_{2,4}}\rangle$\\
$\langle\mathtt{R_{0,5}}$, $\mathtt{S_{1,5}}\rangle$\\
\\
$\langle\mathtt{R_{1,3}}$, $\mathtt{S_{2,7}}\rangle$\\
\end{tabular}
\end{lrbox}




\section{SMT Model}\label{sec:smt}

%The next step, after formally defining the operational semantics of our trace
%language, is to formally define a translation from the trace language into an
%SMT problem that can correctly models the original execution trace.

The novelty of the SMT encoding in this paper is its use of match pairs rather than the state-based or order-based encoding of prior work~\cite{elwakil:padtad10,elwakil:atva10}. The algorithm to create the encoding takes as input a set of possible match pairs and a trace through an MCAPI program with the appropriate assumes and asserts as shown in \figref{fig:trace}. A match pair is the coupling of a receive to a particular send. \figref{fig:smt}(a) is the set of possible match pairs for the program in \figref{fig:mcapi} using our shorthand notation defined in \figref{fig:trace}. The set admits, for example, that $\mathtt{R_{0,2}}$ can be matched with either $\mathtt{S_{1,5}}$ or $\mathtt{S_{2,4}}$. The SMT encoding in this paper asks the SMT solver to resolve the match pairs for the system in such a way that the final values of program variables meet the assumption on control flow but violate some assertion. This resolution by the SMT solver is accomplished by having the solver complete a partial order on events into a total order that determines final match-pair relationships.
\begin{definition}
For a wait operation \texttt{W} in the trace, we create a corresponding variable
$\mathit{event}_\mathtt{W} \in \mathbb{N}$, where $\mathbb{N}$ is the set of natural numbers. This variable will be assigned by the solver to its location in the total order of the program trace.
\label{def:event}
\end{definition}
\begin{definition}
For every send operation $\mathtt{S}$ we create a tuple of variables
$(M,\mathit{event}, e,\mathit{value})$ such that $\mathit{M}$,
$\mathit{event}$, $\mathit{e}$, $\mathit{value}$ $ \in
\mathbb{N}$. $M$ and $\mathit{event}$ will eventually be assigned, by
the SMT solver, the point in the total order of the corresponding matching receive event and the point at which this send appears in the
total order of the same trace. The variables $e$ and $\mathit{value}$ are
assigned the endpoint and actual value contained in the send
operation.
\label{def:snd}
\end{definition}
\begin{definition}
For every receive operation $\mathtt{R}$ we create a tuple of
variables $(M,\mathit{value},\mathit{event},e,\mathit{NW})$ similar to
send, where $\mathit{NW}$ stands for nearest enclosing wait (defined shortly). In the receive case, $M$, $\mathit{value}$ and $\mathit{event}$ will eventually
be assigned, by the SMT solver, the point in the total order of the
corresponding send event with its sent value
and the point at which this receive appears in the total order of the same
trace. The variables $e$ and $\mathit{NW}$ are assigned the endpoint
and the corresponding variable representing the point in the total order of the nearest enclosing wait.
\label{def:rcv}
\end{definition}
The nearest-enclosing wait for a receive witnesses the completion of
the receive indicating that the message is delivered and that all the
previous receives in the same task issued earlier are complete as
well. This constraint is required by the message non-overtaking
property in the MCAPI specification. The intuitive example below shows
that the wait $\mathtt{W{(\&h2)}}$ witnesses the completion of the
receive $\mathtt{R_{0,1}}$ and $\mathtt{R_{0,2}}$ in Task 0. As such,
the wait $\mathtt{W{(\&h2)}}$ is their nearest-enclosing wait. Please
note that the encoding in this paper does not utilize the wait for
sends as part of how it manages the different buffering semantics. Further, this paper assumes that each receive operation has a nearest enclosing wait in the trace.
\[
\begin{array}{l|l}
\;\;\;\;\;\;\;\;\mathtt{Task\ 0}\;\;\;\;\;\;\;\; & \;\;\;\;\;\;\;\; \mathtt{Task\ 1}\;\;\;\;\;\;\;\; \\
\hline
\;\;\;\;\;\;\;\;\mathtt{R_{0,1}(*,\&h1)}\;\;\;\;\;\;\;\; & \;\;\;\;\;\;\;\; \mathtt{S_{1,1}(0,\&h3)}\;\;\;\;\;\;\;\; \\
\;\;\;\;\;\;\;\;\mathtt{R_{0,2}(*,\&h2)}\;\;\;\;\;\;\;\; & \;\;\;\;\;\;\;\; \mathtt{W{(\&h3)}}\;\;\;\;\;\;\;\; \\
\;\;\;\;\;\;\;\;\mathtt{W{(\&h2)}}\;\;\;\;\;\;\;\; & \;\;\;\;\;\;\;\; \mathtt{S_{1,2}(0,\&h4)}\;\;\;\;\;\;\;\; \\
\;\;\;\;\;\;\;\;\mathtt{W{(\&h1)}}\;\;\;\;\;\;\;\; & \;\;\;\;\;\;\;\; \mathtt{W{(\&h4)}}\;\;\;\;\;\;\;\; \\
\end{array}
\]

For convenience, we use $\mathtt{op} \in \{\mathtt{S}, \mathtt{R},
\mathtt{W}\}$ as a subscript to indicate variables associated with different
operations in the trace such as $\mathit{event}_\mathtt{op}$, $M_{\mathtt{op}}$, $e_{\mathtt{op}}$, $value_{\mathtt{op}}$, etc.\footnote{For brevity the presentation omits the subscript and parameter details in the \texttt{S}, \texttt{R}, and \texttt{W} operations, but these details are used to uniquely identify each operation in the figures and examples.} 
\begin{definition}
The \emph{Happens-Before} $(\mathtt{HB})$ relation denoted as $\mathrm{\prec_\mathtt{HB}}$ is a partial order defined over variables. Given any two variables for event order $event_{\mathtt{op}}$ and $event_{\mathtt{op'}}$, $event_{\mathtt{op}}$ $\mathrm{\prec_{\mathtt{HB}}}$ $event_{\mathtt{op'}}$ if and only if $\mathtt{op}$ must complete before $\mathtt{op'}$ in a valid program execution.
\label{def:hb}
\end{definition}
\begin{definition}
For a send $\mathtt{S}=(M_\mathtt{S},\mathit{event}_\mathtt{S}, e_\mathtt{S},\mathit{value}_\mathtt{S})$ and a receive $\mathtt{R}=(M_\mathtt{R},\mathit{value}_\mathtt{R}, \mathit{event}_\mathtt{R}, e_\mathtt{R},\mathit{NW}_\mathtt{R})$, a match pair, $\langle\mathtt{R}, \mathtt{S}\rangle$, is created by adding the constraints
\begin{compactenum}
\item $M_{\mathtt{R}} = event_{\mathtt{S}}$
\item $M_{\mathtt{S}} = event_{\mathtt{R}}$ 
\item $e_{\mathtt{R}} = e_{\mathtt{S}}$
\item $value_{\mathtt{R}} = value_{\mathtt{S}}$ and 
\item $event_{\mathtt{S}}\ \mathrm{\prec_\mathtt{HB}}\ NW_{\mathtt{R}}$
\end{compactenum}
\label{def:match}
\end{definition}
A match-pair only allows the SMT solver to use compatible send/receive
pairs, but more critically, the added ordering in the \texttt{HB}
relation ensures that the send completes before the witness to the
receive to enable the infinite-buffer semantics. There are no other
constraints on a send beyond program order and that only when the
sends target a common endpoint. In such a situation, we use the
\texttt{HB} relation between the sends to prevent message overtaking in
any final total order from the SMT solver.

The infinite-buffer SMT encoding precedes in several stages given a trace and potential match-pairs:
\begin{compactenum}
\item Create all the necessary variables for the sends, receives, waits, and other program variables in the trace and initialize the static fields of the send and receive variables.
\item For each thread, if there are multiple send operations, say $\mathtt{S}$ and $\mathtt{S^\prime}$, from that thread to a common endpoint, $e_\mathtt{S} = e_\mathtt{S^\prime}$, then those sends must follow program order: $\mathit{event}_\mathtt{S}$ $\prec_\mathtt{HB}$ $\mathit{event}_\mathtt{S^\prime}$.
\item For each thread, if there are multiple receive operations, say $\mathtt{R}$ and $\mathtt{R^\prime}$, from that thread to a common endpoint, $e_\mathtt{S} = e_\mathtt{S^\prime}$, then those receives must follow program order: $\mathit{event}_\mathtt{R}$ $\prec_\mathtt{HB}$ $\mathit{event}_\mathtt{R^\prime}$.
\item For every receive \texttt{R} and its nearest enclosing wait \texttt{W}, $\mathit{event}_\mathtt{R}$ $\prec_\mathtt{HB}$ $\mathit{event}_\mathtt{W}$.
\item For any pair of sends $\mathtt{S}$ and $\mathtt{S'}$ on common endpoints, $e_{\mathtt{S}}=e_{\mathtt{S'}}$, such that $\mathit{event}_\mathtt{S}\ \mathrm{\prec_\mathtt{HB}}\ \mathit{event}_\mathtt{S'}$, then those sends must be received in the same order: $M_{\mathtt{S}}\ \mathrm{\prec_{\mathtt{HB}}}\ M_{\mathtt{S'}}$.
\item For every assume representing control flow resolution, add an assert constraint.
\item For every assert representing a checked property, add its negated form as a constraint since our goal is to resolve non-determinism in match-pairs in a way that violates the assertion while following the same control flow.
\item For every match pair $\langle\mathtt{R}, \mathtt{S}\rangle$, add the constraints from $\langle\mathtt{R}, \mathtt{S}\rangle$ in \defref{def:match} to the encoding. If a receive can match to multiple potential sends $\langle\mathtt{R}, \mathtt{S}\rangle$ and $\langle\mathtt{R}, \mathtt{S^\prime}\rangle$, combine the sets of constraints in a disjunction: $\langle\mathtt{R}, \mathtt{S}\rangle$ $\vee$ $\langle\mathtt{R}, \mathtt{S^\prime}\rangle$.
\end{compactenum}
Intuitively, the \texttt{HB} relation asserts that sends and receives
with common endpoints and in the same task follow program order
(stages 2 and 3); a receive operation must happen before its
nearest-enclosing wait (stage 4); sends must be received in the same
order they are sent (stage 5); control flow is enforced and assert
violations are detected (stages 6 and 7); and finally, only one
match-pair can be resolved for any given receive (stage 8).

As a further clarification on stage 5, consider a simple example
below that sends two messages from a Task 0 to Task 1,
\[
\begin{array}{l|l}
\;\;\;\;\;\;\;\;\mathtt{Task\ 0}\;\;\;\;\;\;\;\; & \;\;\;\;\;\;\;\; \mathtt{Task\ 1}\;\;\;\;\;\;\;\; \\
\hline
\;\;\;\;\;\;\;\;\mathtt{S_{0,1}(1,\&h1)}\;\;\;\;\;\;\;\; & \;\;\;\;\;\;\;\; \mathtt{R_{1,1}(*,\&h3)}\;\;\;\;\;\;\;\; \\
\;\;\;\;\;\;\;\;\mathtt{S_{0,2}(1,\&h2)}\;\;\;\;\;\;\;\; & \;\;\;\;\;\;\;\; \mathtt{R_{1,2}(*,\&h4)}\;\;\;\;\;\;\;\; \\
\;\;\;\;\;\;\;\;\mathtt{W(\&h1)}\;\;\;\;\;\;\;\; & \;\;\;\;\;\;\;\; \mathtt{W(\&h3)}\;\;\;\;\;\;\;\; \\
\;\;\;\;\;\;\;\;\mathtt{W(\&h2)}\;\;\;\;\;\;\;\; & \;\;\;\;\;\;\;\; \mathtt{W(\&h4)}\;\;\;\;\;\;\;\; \\
\end{array}
\]
The \texttt{M} variables in the send tuples will be assigned to the
order tracking events for $\mathtt{R_{1,1}}$ and $\mathtt{R_{1,2}}$ by the match-pairs selected by the SMT solver. The constraints added in stage 5 force the send events to be received in
program order using the \texttt{HB} relation which for our simple
example yields
$\cfgnt{M}_\mathtt{S_{0,1}}\ \mathrm{\prec_\mathtt{HB}}\ \cfgnt{M}_\mathtt{S_{0,2}}$.

For zero-buffer semantics, we further constrain the encoding to
preclude orderings that can only occur with buffering. For example in
\figref{fig:mcapi}, to prevent $\mathtt{R_{1,3}}$ from being matched
with $\mathtt{S_{2,6}}$ before $\mathtt{R_{0,2}}$ is matched with
$\mathtt{S_{2,4}}$, add
$\mathtt{W(\&h1)}\ \mathrm{\prec_\mathtt{HB}}\ \mathtt{S_{2,6}}$.

\begin{figure}
\begin{center}
\setlength{\tabcolsep}{20pt}
\begin{tabular}[t]{cc}
\scalebox{0.7}{\usebox{\boxMP}} &
\scalebox{0.7}{\usebox{\boxSMTc}} \\\\
(a) & (b)
\end{tabular}
\end{center}
\caption{A match pair set and SMT encoding of the system in \figref{fig:mcapi}.
(a) The match pairs based on endpoints. (b) The SMT encoding where $\mathtt{\prec_\mathtt{HB}}$ creates a
\emph{Happens-Before} constraint, a pair surrounded by $\langle$ and $\rangle$ creates a match pair constraint, and $\mathtt{assume}$ and $\mathtt{assert}$ creates an assume and assert, respectively.}
\label{fig:smt}
\end{figure}

\figref{fig:smt}(b) is the final encoding for the trace which is
explained here. The encoding is divided into three sections:
\textit{SMT = \{defs constraints match\}}. Stage 1 is not shown
because the definitions are not novel to our solution; suffice to say
that variables are created as defined. The constraints sections is created by stages 2-7. Lines \texttt{00} - \texttt{10} reflect program orders in the trace. Lines \texttt{11} and \texttt{12} for the assume and assert commands of the original execution trace in \figref{fig:trace}. The first assert on line \texttt{11} (line \texttt{14} at trace 1) prevents the SMT solver from finding solutions that are not consistent with control flow which requires ``$b \ge 0$". The second assert on line \texttt{12} is negated as the goal is to find schedules that violate the property. Finally, stage 8 generates the $\mathit{match}$ ``area" of the SMT encoding. In particular, we send the set of match pairs in \figref{fig:smt}(a) to the algorithm and generate each match pair and collect those on the same receive into a disjunction on line \texttt{13} - \texttt{15}.

Other than the basic structure of the SMT encoding, we use the function
\[\mathrm{ANS}(\mathit{smt}) \mapsto \{\mathrm{\cfgt{SAT}},\mathrm{\cfgt{UNSAT}}\} \]
to return the solution of an SMT problem, such that $\mathrm{\cfgt{SAT}}$ represents a satisfiable solution that finds a trace of the MCAPI program execution that violates the user defined correctness property, and $\mathrm{\cfgt{UNSAT}}$ represents an unsatisfiable solution indicating that all possible execution traces either meet the correctness property in the same control flow, or follow a different control flow. Note that $\mathrm{\cfgt{UNSAT}}$ and $\mathrm{\cfgt{SAT}}$ are ordered such that $\mathrm{\cfgt{UNSAT}} < \mathrm{\cfgt{SAT}}$.

The SMT encoding defined above is used to capture the non-deterministic behaviors of an MCAPI program execution by giving a complete set of match pairs. As we discussed in the previous section, the MCAPI program in \figref{fig:mcapi} contains two outcomes of execution as defined in the MCAPI specification under the infinite-buffer semantics. The SMT encoding we present in \figref{fig:smt}(b) captures both execution traces, since the set of match pairs in \figref{fig:smt}(a) is a complete set where all matches that can occur in the real execution for our running example in \figref{fig:mcapi} are included, and all matches that cannot occur in the real execution are not included. Further, the following theorem states that we can over-approximate the true set of match pairs and still prove correctness. If there is no error with the over-approximated set, then there is no error arising from non-determinism in the runtime on that program execution. If there is an error from the over-approximated set, that error is also guaranteed to be a real error in the program runtime. Note that two SMT problems $\smt_{\alpha}$ and $\smt_{\beta}$ in the following theorem have identical $\mathit{defs}$ and $\mathit{constraints}$ sets, and differ only by match set.
\\
\\
\textbf{Theorem 1.}
The relation for the solutions of two SMT problems $\smt_{\alpha}$ $= (\mathit{defs}$ $\mathit{constraints}$ $\mathit{match}_{\alpha})$ and $\smt_{\beta}$ $= (\mathit{defs}$ $\mathit{constraints}$ $\mathit{match_{\beta}})$ is,
\[\mathrm{ANS}(\smt_{\alpha}) \leq \mathrm{ANS}(\smt_{\beta})\]
where $\mathit{set(match_{\alpha})} \subseteq \mathit{set(match_{\beta})}$. Note that $\mathit{set(match)}$ represents the input set of match pairs in the $\mathit{match}$ field.
\label{thm:1}
\begin{figure}
\begin{center}
\setlength{\tabcolsep}{3pt}
\begin{tabular}[c]{cc}
%\scalebox{0.7}{\usebox{\boxMP}} &
\scalebox{0.7}{\usebox{\boxSMTa}} &
\scalebox{0.7}{\usebox{\boxSMTb}} \\\\
%\scalebox{0.7}{\usebox{\boxSMTc}} \\
(a) & (b)
\end{tabular}
\end{center}
\caption{Two SMT encodings of the system in \figref{fig:mcapi}.
(a) The SMT encoding for Trace 1 in \figref{fig:trace}. (b) The SMT encoding for Trace 2 in \figref{fig:trace}.}
\label{fig:smt_trace}
\end{figure}
\\
\textbf{Proof Sketch.}
Consider the MCAPI program in \figref{fig:mcapi} as an example.
\figref{fig:smt_trace}(a) and (b) are two different SMT encodings for our running
example in \figref{fig:mcapi} generated from different sets of possible match pairs, such that \figref{fig:smt_trace}(a) encodes trace 1 in \figref{fig:trace} and \figref{fig:smt_trace}(b) encodes trace 2 in \figref{fig:trace}. By solving the encodings in \figref{fig:smt_trace}(a) and (b) for trace 1 and 2 in \figref{fig:trace} respectively, we get an unsatisfiable solution for \figref{fig:smt_trace}(a), and a satisfiable solution for \figref{fig:smt_trace}(b). As we discussed in Section 2, trace 1 is an execution trace without failure of the assertion, and trace 2 is the one that fails the assertion. %We can see that the encodings in \figref{fig:smt_trace} (a) and (b) correctly encodes trace 1 and 2 in \figref{fig:trace}, respectively.
Compare both encodings with that in \figref{fig:smt}(b), we find that the fields $\mathit{defs}$ and $\mathit{constraints}$ are identical except for the set of match pairs. In particular, the input set of match pairs of either \figref{fig:smt_trace}(a) or (b) is the subset of that in \figref{fig:smt}(b). As discussed above, the encoding in \figref{fig:smt}(b) captures the non-deterministic behavior of our running program in \figref{fig:mcapi}, which encodes trace 1 and 2 in \figref{fig:trace} into one single SMT problem. Thus, an SMT solver will return a satisfiable solution for the encoding in \figref{fig:smt}(b). Thus, $\mathrm{ANS}$ on the encoding in \figref{fig:smt}(b) is greater than or equal to the $\mathrm{ANS}$ on the encoding in either \figref{fig:smt_trace}(a) or (b). In other words, adding more match pairs can only move the $\mathrm{ANS}$ from $\mathrm{\cfgt{UNSAT}}$ to $\mathrm{\cfgt{SAT}}$.

The formal proof of Theorem 1 is in the long version of our paper at ``http://students.cs.byu.edu/$\sim$yhuang2/downloads\\/paper.pdf". The proof defines a formal operational semantics given by a term rewriting system using a \textit{CESK}\footnote{The \textit{CESK} machine state is represented with a \textbf{C}ontrol string, \textbf{E}nvironment, \textbf{S}tore, and \textbf{K}ontinuation.} style machine only the machine is augmented to include additional structure for modeling message passing. The operational framework defines how to execute a program, following the specified trace, and defines when that execution is a success (causes no assertion violation), a failure (causes an assertion violation), infeasible (causes an assume to not hold), or an error (execution is not allowed by the MCAPI semantics.). Further, the machine generates the terms of the SMT encoding as it rewrites the machine states. The proof defines a combination operator and shows that several SMT encodings can be combined such that the combined SMT encoding returns ``SAT" if one of those encodings has a satisfiable solution.  As such, Theorem 1 is formally proved by applying the combination operator for $\smt_{\alpha}$ and $\smt_{\beta}$.

Assume $\mathit{set(match_{\alpha})}$ and $\mathit{set(match_{\beta})}$ in the theorem above a complete set of match pairs and an over-approximated set, respectively, we can further prove that $\mathrm{ANS}(\smt_{\alpha})$ $ = $ $\mathrm{ANS}(\smt_{\beta})$ by giving the following theorem. Note that a match pair $\langle \mathtt{R},\mathtt{S}\rangle \in \mathit{set(match_{\beta})}/\mathit{set(match_{\alpha})}$ is called ``bogus", since it cannot exist in a real execution of the program.
\\
\\
\textbf{Theorem 2.}
Any match pair $\langle \mathtt{R}, \mathtt{S}\rangle$ used in a satisfying assignment of an SMT encoding $\smt$ is a valid match pair and reflects an actual possible MCAPI program execution.
\\
\textbf{Proof.}
Proof by contradiction. Assume that $\langle \mathtt{R}, \mathtt{S}\rangle$ is a ``bogus" match pair that causes $\mathrm{ANS}(\smt) = \cfgt{SAT}$. Since $\langle \mathtt{R}, \mathtt{S}\rangle$ is not a valid match pair, match $\mathtt{R}$ and $\mathtt{S}$ requires program order, message non-overtaking, or no-multiple match to be violated. In other words, the $\mathtt{HB}$ constraints encoded in $\smt$ are not satisfied. Based on the fact above, the answer of $\smt$ is $\cfgt{UNSAT}$ and it contradicts the previous hypothesis. Thus, $\langle \mathtt{R}, \mathtt{S}\rangle$ is a valid match pair in $\smt$ and reflects an actual possible MCAPI program execution. $\Box$


By proving Theorem 2, we infer that a ``bogus" match pair can only cause an unsatisfying assignment of an SMT problem. Further, given that $\mathit{set(match_{\alpha})}$ and $\mathit{set(match_{\beta})}$ reflect a complete set and a over-approximated set respectively, the answers of $\smt_{\alpha}$ and $\smt_{\beta}$ discussed above are equal since any ``bogus" match pair involved in $\smt_{\beta}$ is only used in unsatisfying assignments.











