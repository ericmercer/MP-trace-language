\newcommand{\defs}{\ensuremath{\mathrm{defs}}}
\newcommand{\any}{\ensuremath{\mathrm{any}}}
\newcommand{\negate}{\ensuremath{\mathrm{not}}}
\newcommand{\define}{\ensuremath{\mathrm{define}}}
\newcommand{\andd}{\ensuremath{\mathrm{and}}}
\newcommand{\select}{\ensuremath{\mathrm{select}}}
\newcommand{\HB}{\ensuremath{\mathrm{HB}}}
\newcommand{\last}{\ensuremath{\mathrm{last}}}
\newcommand{\MATCH}{\ensuremath{\mathrm{MATCH}}}



\newsavebox{\boxSMTa}
\begin{lrbox}{\boxSMTa}
\begin{minipage}[c]{0.4\linewidth}
\begin{alltt}
(HB rcvA (wait rcvA))
(HB (wait rcvA) rcvB)
(HB rcvB (wait rcvB))
(HB (wait rcvB) assume)
(HB assume assert)
(HB rcvC (wait rcvC))
(HB (wait rcvC) snd3)
(HB snd3 (wait snd3))
(HB snd1 (wait snd1))
(HB (wait snd1) snd2)
(HB snd2 (wait snd2))

(MATCH rcvA snd1)
(MATCH rcvB snd3)
(MATCH rcvC snd2)

(assume (> b 0))
(assert (not (= a 4)))
\end{alltt}
\end{minipage}
\end{lrbox}

\newsavebox{\boxSMTb}
\begin{lrbox}{\boxSMTb}
\begin{minipage}[c]{0.4\linewidth}
\begin{alltt}
(HB rcvA (wait rcvA))
(HB (wait rcvA) rcvB)
(HB rcvB (wait rcvB))
(HB (wait rcvB) assume)
(HB assume assert)
(HB rcvC (wait rcvC))
(HB (wait rcvC) snd3)
(HB snd3 (wait snd3))
(HB snd1 (wait snd1))
(HB (wait snd1) snd2)
(HB snd2 (wait snd2))

(MATCH rcvA snd3)
(MATCH rcvB snd1)
(MATCH rcvC snd2)

(assume (> b 0))
(assert (not (= a 4)))
\end{alltt}
\end{minipage}
\end{lrbox}

\newsavebox{\boxSMTc}
\begin{lrbox}{\boxSMTc}
\begin{minipage}[c]{0.4\linewidth}
\begin{alltt}
(HB rcvA (wait rcvA))
(HB (wait rcvA) rcvB)
(HB rcvB (wait rcvB))
(HB (wait rcvB) assume)
(HB assume assert)
(HB rcvC (wait rcvC))
(HB (wait rcvC) snd3)
(HB snd3 (wait snd3))
(HB snd1 (wait snd1))
(HB (wait snd1) snd2)
(HB snd2 (wait snd2))

(or (MATCH rcvA snd1)
    (MATCH rcvA snd3))
(or (MATCH rcvB snd1)
    (MATCH rcvB snd3))
(MATCH rcvC snd2)

(ne rcvA rcvB)

(assume (> b 0))
(assert (not (= a 4)))


\end{alltt}
\end{minipage}
\end{lrbox}


%%---------------------------------------------------------------------------
%%% Start --- SMT Machine reductions
\begin{figure}
\mprset{flushleft}
\scalebox{0.75}{
\begin{mathpar}
\and
%  (--> (h eta aid-map ep-send-calls pending-r
%           (thread_0 ... ([ploc_0 cmd_0] [ploc_1 cmd_1] [ploc_2 cmd_2] ...) thread_2 ...)
%           (ploc_0 trace-entry_1 ...) status smt)
%        (h_pr eta_pr aid-map_pr ep-send-calls_pr pending-r_pr
%              (thread_0 ... ([ploc_1 cmd_1] [ploc_2 cmd_2] ...) thread_2 ...)
%              (trace-entry_1 ...) status_pr smt_prpr)
%        "Machine Step"
%        (where (h_pr eta_pr aid-map_pr ep-send-calls_pr pending-r_pr
%                     e status_pr ret smt_pr)
%               ,(apply-reduction-relation** expr-reductions
%                                            (term (h eta aid-map ep-send-calls
%                                                     pending-r
%                                                     cmd_0 status ret smt))))
%        (where smt_prpr (add-po smt_pr ploc_0 ploc_1)))
\inferrule[Machine Step]{
  (\epsnd~\qset~\movebot~\status)~\reduceK{q}~
  (\epsnd_\mathit{p0}~\qset_\mathit{p0}~\movebot_p~\status_\mathit{p0}) \\\\
  (h~\eta~\aidmap~\epsnd_\mathit{p0}~\eprcv~\qset_\mathit{p0}~\cmd_0~\status_\mathit{p0}~\ret~\smt)~\reduceK{e}~
  (h_p~\eta_p~\aidmap_p~\epsnd_\mathit{p1}~\eprcv_p~\qset_\mathit{p1}~e~\status_\mathit{p1}~\ret~\smt_\mathit{p0}) \\
  \smt_\mathit{p1} =  \mathrm{addHB}(\smt_\mathit{p0}~\ploc_0~\ploc_1)
}{
  (h~\eta~\aidmap~\epsnd~\eprcv~\qset~
           (\thread_0~\ldots~
            ([\ploc_0~\cmd_0]~[\ploc_1~\cmd_1]~[\ploc_2~\cmd_2]~\ldots)~\thread_2~\ldots)~
           ([\ploc_0~\movebot]~\traceentry_1~\ldots)~\status) \reduce{m} \\\\
  (h_p~\eta_p~\aidmap_p~\epsnd_\mathit{p1}~\eprcv_p~\qset_\mathit{p1}~
              (\thread_0~\ldots~([\ploc_1~\cmd_1]~[\ploc_2~\cmd_2]~\dots)~\thread_2~\dots)~
              (\traceentry_1~\dots)~\status_\mathit{p1})
}
\end{mathpar}}
\caption{Machine Reductions to build the SMT model of a trace language program ($\reduce{\mathrm{m-smt}}$)}
\label{fig:machine}
\end{figure}


\begin{figure}
\scalebox{0.75}{%
\mprset{flushleft}
\begin{mathpar}
%   (--> (h eta aid-map pending-s pending-r  (assume e) status k
%           (defs (any ...)))
%        (h eta aid-map pending-s pending-r  e status (assume * -> k )
%           (defs (e any ...)))
%        "Assume Pull Out Expr")
\inferrule[Assume Expressions Evaluation]{}{
  (h\ \eta\ \aidmap\ \epsnd\ \eprcv\ (\assume\ e)\ \status\ k\ (\defs\ (\any\ \ldots))) \reduce{e} 
  (h\ \eta\ \aidmap\ \epsnd\ \eprcv\ e\ \status\ (\assume\ * \rightarrow k)\ (\defs\ (e\ \any\ \ldots)))
}
\and
%   ;;Negate expression and add it to the SMT assertions.
%   (--> (h eta aid-map pending-s pending-r  (assert e) status k
%           (defs (any ...)))
%        (h eta aid-map pending-s pending-r  e status (assert * -> k)
%           (defs ((not e) any ...)))
%        "Assert Pull Out Expr")
\inferrule[Assert Expressions Evaluation]{}{
  (h\ \eta\ \aidmap\ \epsnd\ \eprcv\ (\assert\ e)\ \status\ k\ (\defs\ (\any\ \ldots))) \reduce{e} 
  (h\ \eta\ \aidmap\ \epsnd\ \eprcv\ e\ \status\ (\assert\ * \rightarrow k)\ (\defs\ ((\negate\ e)\ \any\ \ldots)))
}
\and
%   (--> (h eta aid-map pending-s pending-r  (x := e) status k
%           (defs (any ...)))
%        (h eta aid-map pending-s pending-r  e status (x := * -> k)
%           (defs ((= x e) any ...)))
%        "Assign Pull Out Expr")
\inferrule[Assign Expressions Evaluation]{}{
  (h\ \eta\ \aidmap\ \epsnd\ \eprcv\ (x\ :=\ e)\ \status\ k\ (\defs\ (\any\ \ldots))) \reduce{e} 
  (h\ \eta\ \aidmap\ \epsnd\ \eprcv\ e\ \status\ (x\ :=\ * \rightarrow k)\ (\defs\ ((=\ x\ e)\ \any\ \ldots)))
}
\and
%   ;;Adds sendi cmd to aid-map and pending-s. - VALIDATED
%   (--> (h eta ([aid_x ep_x] ...) pending-s pending-r q-set
%           (sendi aid src dst x ploc number) status k ((any_d ...) (any_a ...)))
%        (h eta ([aid src] [aid_x ep_x] ...) pending-s_pr pending-r q-set
%           true status k ((any_0 any_d ...) (any_1 any_a ...)))
%        "Sendi Cmd x -> v"
%        (where v (h-lookup h (eta-lookup eta x)))
%        (where any_0 (define aid :: send))
%        (where any_1 (make-send aid src dst x ploc number))
%        (where pending-s_pr (add-send pending-s [aid -> src dst v])))
\inferrule[Sndi Command]{
  ([\aid_1\ v_1]\ \ldots) = \epsnd(\dst)(\src) \\ \epsnd_p = [\epsnd \mid \dst \mapsto [\epsnd(\dst) \mid \src \mapsto ([\aid_0\ h(\eta(x))]\ [\aid_1\ v_1]\ \ldots)]] \\
  \any_0 = (\define\ \aid\ ::\ send) \\ \any_1 = (\andd\ (=\ (\select\ \aid\ ep)\ \dst)\ (=\ (\select\ \aid\ value)\ x))
}{
  (h\ \eta\ ([\aid\ \ep]\ \ldots)\ \epsnd\ \eprcv\ \qset\
     (\sendi\ \aid_0\ \src\ \dst\ x)\ \status\ k\ ((\any_d\ \ldots)\ (\any_a\ \ldots))) \reduce{e} \\
  (h\ \eta\ ([\aid_0\ \src]\ [\aid\ \ep]\ \ldots)\ \epsnd_p\ \eprcv\ \qset\ \true\ \status\ k\ ((\any_0\ \any_d\ \ldots)\ (\any_1\ \any_a\ \ldots)))
}
\and
%   ;;Adds recvi cmd to aid-map and pending-r. - VALIDATED
%   (--> (h eta ([aid_x ep_x] ...) pending-s pending-r q-set
%           (recvi aid dst x ploc) status k ((any_d ...) (any_a ...)))
%        (h eta ([aid dst] [aid_x ep_x] ...) pending-s pending-r_pr q-set
%           true status k ((any_0 any_d ...) (any_1 any_a ...)))
%        "Recvi Cmd0"
%        (where pending-r_pr (add-recv pending-r [aid -> dst x]))
%        (where any_0 (define aid :: recv))
%        (where any_1 (make-recv aid dst x ploc)))
\inferrule[Rcvi Command]{
  ([\aid_1\ x_1]\ \ldots) = \eprcv(\dst) \\
   \eprcv_p = [\eprcv \mid \dst \mapsto ([\aid_0\ x_0]\ [\aid_1\ x_1]\ \ldots)] \\
   \any_0 = (\define\ \aid\ ::\ recv) \\ \any_1 = (\andd\ (=\ (\select\ \aid\ ep)\ \dst)\ (=\ (\select\ \aid\ var)\ x_0))
}{
  (h\ \eta\ ([\aid\ \ep]\ \ldots)\ \epsnd\ \eprcv\ \qset\ (\recvi\ \aid_0\ \dst\ x_0)\ \status\ k\ ((\any_d\ \ldots)\ (\any_a\ \ldots))) \reduce{e} \\
  (h\ \eta\ ([\aid_0\ \dst]\ [\aid\ \ep]\ \ldots)\ \epsnd\ \eprcv_p\ \qset\ \true\ \status\ k\ ((\any_0\ \any_d\ \ldots)\ (\any_1\ \any_a\ \ldots)))
}
\and
   %(--> (h eta aid-map pending-s pending-r q-set (wait aid_r)
%           status k (defs (any_a ...)))
%        (h_pr eta aid-map_pr pending-s pending-r_pr q-set_pr true
%              status_pr k
%              (defs ((HB (select aid-��_s-last match_po) (select aid_s match_po))
%                    (MATCH aid_r0 aid_s0) any_a ...)))
%        "Recvi Wait Cmd"
%            ;; get recv and corresponding send command, mark each send in front of the destination send (by set variable "mark" to 1)
%        (where true (find-recv pending-r aid_r) )
%        (where (aid-map_0 dst_r) (get-ep aid-map aid_r))
%        (where (h_pr aid_s pending-r_pr q-set_pr status_pr)
%               (get-mark-remove h eta pending-r q-set dst_r aid_r status))
%        (where (aid-map_pr src_s) (get-ep aid-map_0 aid_s))
%        )
%   ))
\inferrule[Wait (rcvi) Command]{
    true = \findrecv(\eprcv, \aid_r) \\ (\aidmap_0\ \dst) = \getep(\aidmap, \aid_r) \\
    (h_p\ \aid_s\ \eprcv_p\ \qset_p\ \status_p) = \getmarkremove(h, \eta, \eprcv, \qset, \dst, \aid_r, \status) \\
    (\aidmap_p\ \src) = \getep(\aidmap_0, \aid_s)
}{
    (h\ \eta\ \aidmap\ \epsnd\ \eprcv\ \qset\ (\wait\ \aid_r)\ \status\ k\ (\defs\ (\any_a\ \ldots))) \reduce{e} \\
    (h_p\ \eta \aidmap_p\ \epsnd\ \eprcv_p\ \qset_p\ true\ \status_p\ k\ (\defs\ ((\HB\ (\select\ \last(\aid_s)\ match_{po})\ (\select\ \aid_s\ match_{po}))\
                    (\MATCH\ \aid_r\ \aid_s)\ \any_a\ \ldots)))
}
\end{mathpar}}
\caption{Expression Reductions to build the SMT model of a trace language program ($\reduce{\mathrm{e-smt}}$)}
\label{fig:expression}
\end{figure}
%%% END --- SMT Machine reductions
%%---------------------------------------------------------------------------

\section{SMT Model}\label{sec:smt}

The next step, after formally defining the operational semantics of our trace
language, was to formally define a translation from the trace language into an
SMT problem that would correctly model the original execution trace. We
achieved this by building on the same framework that we constructed to execute
the semantics that we defined for our trace language. In addition to executing
the semantic meaning of the trace language, we modified our system so that it
would also produce a correlating SMT problem for a given execution trace.

The SMT problem that we form is made of various free variables, bound
variables, and logical asserations. For example, we could define an SMT problem
that includes the bound variable A and B with their assignments of A=1 and B=2,
a free variable C, and an assertion that A + B = C. Feeding this into an SMT
solver would produce a result showing that the problem is satisfiable if C=3.
Alternatively, we could use alter the problem so that B is also a free
variable. Feeding this problem into an SMT solver would produce any number of
results showing values for B and C, such that B = C-1.

\begin{figure}
\begin{center}
\setlength{\tabcolsep}{7pt}
\begin{tabular}[c]{ccc}
\scalebox{0.68}{\usebox{\boxSMTa}} &
\scalebox{0.68}{\usebox{\boxSMTb}} &
\scalebox{0.68}{\usebox{\boxSMTc}} \\
(a) & (b) & (c)
\end{tabular}
\end{center}

\caption{SMT problems. (a) SMT problem based on the first trace. (b) SMT
problem based on a second trace. (c) SMT problem built from the preceding two
problems.}

\label{fig:smts}
\end{figure}

\figref{fig:smts} contains three SMT problems. The first two are created from
two different executions of the same trace. The third SMT problem is a
combination of the first two. As you can see in the figure, the combined SMT
problem is very similar to the first two. The only difference is the way in
which the possible match pairs are described. The first two SMT problems
represent specific traces, so each match pair is explicitly defined. The third
SMT problem allows behaviors from either of the first two problems, so it
defines its match pairs such that $\mathrm{rcvA}$ and $\mathrm{rcvB}$ can match
with either $\mathrm{snd1}$ or $\mathrm{snd3}$. This also requires the addition
of the $\mathrm{ne}$ function that asserts that $\mathrm{rcvA}$ and
$\mathrm{rcvB}$ do not match with the same send operation. We will now explain
the various data structures and functions from the SMT problems in more depth.

\begin{equation}
\mathrm{HB}(a,b) \equiv \left\{ \begin{array}{rl}
  \mathit{true} & \mathrm{if}\ a < b \\
  \mathit{false} & \mathrm{otherwise}\
\end{array}
\right .
\end{equation}

The HB function represents a happens-before relationship between two events. It
takes two events $a$ and $b$ as parameters, such that $a,b \in \mathbb{N}$. This
function creates a constraint such that the first event must be less than the
second event, indicating that the first event occurs before the second event.
This function is used to assert the program order of statements within the same
thead, and also to assert that, for any matched pair, that the send operation
occurs before the receive.

\begin{equation}
\mathrm{MATCH}(R,S) \equiv \left\{
\begin{array}{rl}
  \mathit{true} & \mathrm{if}\ M = I~\wedge \\
  & E_r = E_s~\wedge \\
  & V_r = V_s~\wedge \\
  & \mathrm{HB}(T_s,T_r) \\
  \mathit{false} & \mathrm{otherwise}\
\end{array}
\right . \\
\end{equation}

The $\mathrm{MATCH}$ function takes two four-tuples as parameters. The first tuple
represents the a receive operation, and the second represents a send operation.
The two tuples are evalueted to see whether or not the two operations could
conceivably be matched.  The receive tuple, $R$, is defined as $R =
(M,E_r,V_r,T_r)$ such that $M,E_r,V_r,T_r \in \mathbb{N}$. $M$ represents a
free variable that is used to define a match pair. The variable must be equal
to the corresponding value in a send tuple in order for the two operations to
be matched. $E_r$ represents the end point of the receive operation. $V_r$
reprensents the the value received by the receive operation. $T_r$ represents
the program order of the receive operation.

The second tuple $S$ represents a send operation that is defined as $S =
(I,E_r,V_r,T_r)$ such that $I,E_r,V_r,T_r \in \mathbb{N}$. $I$ is a unique
assigned value that identifies each send operation. $E_r$ represents that end
point that is being sent a message. $V_r$ contains the value being sent, and
$T_r$ represents the program order of the receive operation.

The MATCH function creates a number of constraints that represent the pairing
of the specified send and receive operations. This function asserts that the
destination endpoint of the send operation is the same as the endpoint used by
the receive operation, that the value sent by the send operation is the same
value received by the receive operation, that the send operation occurs before
the receive operation, and that the ``match'' value in the recveive tuple is
equal to the ``id'' value in the send tuple, which indicates that the receiver
operation is paired with only this send operation.

\begin{equation}
\mathrm{ne}(a,b) \equiv \left\{ \begin{array}{rl}
  \mathit{true} & \mathrm{if}\ M_a \not = M_b \\
  \mathit{false} & \mathrm{otherwise}\
\end{array}
\right .
\end{equation}

The ne function is used to assert that no two receive operations are matched
with the same send operation. The parameters for this function are two receive
tuples. An assertion is made that the values of their ``match'' fields, $M$,
are not equal. This shows that they are paired with two different send
operations.

The first five statments in each SMT problem in \figref{fig:smts} uses the
$\mathrm{HB}$ function to assert the program order of each thread as defined by
the CTP. The last two lines of each problem contain $\mathrm{assume}$ and
$\mathrm{assert}$ statments. The $\mathrm{assume}$ statement is created by
asserting any branch conditions that we encountered in the trace. As was stated
earlier, this prevents the SMT solver from finding any solutions that are not
consistent with the program's control flow. The $\mathrm{assert}$ statement is
created by negating any assertions that were made in the original program. This
creates an SMT problem that will only be satisfiable if an assertion violation
can be found.

The rest of the SMT problems deals with defining match pairs. In the first two
SMT problems the match pairs are taken straight from their traces. In order to
combine these possible match pairs and create the third SMT problem, we can
find the union of all send operations that could possibly match with each
receive operation. For each receive operation, we assert a MATCH between any
one of the send operations in the set of all operations that the receive could
be matched with. In order to make sure that no two receive operations are
matched with the same send operation, we also create an $\mathrm{ne}$ function
call for each pair of receive statments who have a non-empty intersection
between their sets of possible matching send operations.

The evaluation syntax for the machine reductions to build the SMT
model of a trace language program are largely those of the regular machine
reductions with a few changes below
\[
\cfgstart
\cfgrule{mstate}{\lp\cfgnt{h}\ \cfgnt{\ensuremath{\eta}}\ \cfgnt{\aidmap}\
  \cfgnt{\epsnd}\ \cfgnt{\eprcv}\ \cfgnt{ctp}\ \cfgnt{\trace}\ \cfgnt{\status}\ \cfgnt{k}\ \smt\rp}
%  (expr-state (h eta aid-map pending-s pending-r e status k smt))
\cfgrule{estate}{\lp\cfgnt{h}\ \cfgnt{\ensuremath{\eta}}\ \cfgnt{\aidmap}\
  \cfgnt{\epsnd}\ \cfgnt{\eprcv}\ \cfgnt{e}\ \cfgnt{\status}\ \cfgnt{k}\ \smt\rp}
%  (smt (defs asrts))
\cfgrule{\smt}{\lp\cfgnt{defs}~\cfgnt{constraints}\rp}
%  (defs (any ...))
\cfgrule{defs}{\lp\cfgnt{any}~$\ldots$\rp}
%  (constraints (any ...))
\cfgrule{constraints}{\lp\cfgnt{any}~$\ldots$\rp}
\cfgend
\]
The new syntax adds an \textit{smt} member to the \textit{mstate}
and \textit{estate} where the ``\cfgnt{any}'' term in \textit{defs}
and \textit{constraints} matches any structure.  The lists will
be filled with definitions, HB entries, MATCH entries, \texttt{ne}
entries, etc. as defined by the SMT machine reductions.

There are several support functions for the SMT machine reductions.  The function $\mathrm{addHB}(((\mathit{any}_d ...)\ (\mathit{any}_c ...))\ \ploc_0\ \ploc_1)$ adds program location to the definition list and adds a happens before relation
\[
 \equiv \\
   (((\mathtt{define \ploc_0 :: int}) \mathit{any}_d ...)((\mathtt{HB \ploc_0 \ploc_1}) \mathit{any}_c ...))
\]

\begin{definition}
A machine state $(h\ \eta\ \aidmap\ \epsnd\ \eprcv\ \cfgnt{ctp}\ \trace\ \status\ k)$ is smt-enabled if it is well formed.  The SMT problem is taken from the final SMT reduced state:
\[\begin{array}{l}
        (h\ \eta\ \aidmap\ \epsnd\ \eprcv\ \cfgnt{ctp}\ \trace\ \status\ k\ (()()))  \reduceK{\mathrm{m-smt}} \\
 (h_p\ \eta_p\ \aidmap_p\ \epsnd_p\ \eprcv_p\ \cfgnt{ctp}_p\ \trace_p\ \status_p\ k_p\ \mathrm{smt})
 \end{array}\]
For convenience, we define the function
$\mathrm{smt}($m$) \mapsto \{\cfgt{SAT},\cfgt{UNSAT}\}$
to return the status of the SMT problem in the final state of $m$.
\end{definition}

\begin{lemma}
For a well-formed machine state $m$
\[
\mathrm{status}(m) = \cfgt{failure} \iff \mathrm{smt}(m) = \cfgt{satisfiable}
\]
\end{lemma}
\begin{proof}
The Proof relies on the operational semantics in that each rule that adds
something to the SMT is trivially shown to be equivalent to the
trace-language action.  If the trace-language only represents a single
trace, then by implication, the SMT problem only encodes a single
trace.  Key is the one-to-one mapping between the trace-language
operations and the SMT constructs.
\end{proof}

\begin{theorem}
My Lemma
\end{theorem}
\begin{proof}
Proof
\end{proof}

\subsection{Outline}

SMT model is created from an extension to the operational semantics of the
trace language.  The semantics generate an SMT for a single trace.  The key
observation is that the SMT does not change for different traces on the same
CTP except for the specified match pairs.  This observation leads to the main
result that we can over-approximate the match pair sets and use the
trance-language in a refinement loop.

Add a figure showing the encoding of two traces for the example system.   Put the two traces next to each other.  Use the ones from LaDefense.pptx
\begin{itemize}
\item Define upfront MATCH, HB, and the data structure definitions (anything boiler plate)
\item Focusing on one trace, give the new reductions for assert and assume referencing directly the line number in the trace where these statements appear.
\item Give the new reductions for WAITI-rcv to define a match pair.  Again reference the line number in the figure for clarity.
\item Add the full semantics to the appendix and give the reference here.
\item Lemma 1: we can encode a single trace.
\item Examine the second trace.  Note the only difference is seen in the MATCH pairs.
\item Define the combinator operation that combines two SMT encodings for a common CTP
\item Lemma 2: we can encode multiple traces into a single SMT problem.  Proof by induction.  Base case is a single trace.  Trivially established by Lemma 1.  Induction is assume you have correctly encoded n-traces, and you are adding a new trace (i.e., a new set of match pairs) to the encoding using your combinator operation.
\item Suppose we an answer function that given a MP set and a CTP returns either success or failure with a trace.
\item Define the main theorem for the paper (slide 68 in LaDefense.pptx).  Generalize to arbitrary sets.  Proof by induction.  Base case is a single trace in $\mathrm{mp}_1$ so $\mathrm{mp}_1 \subseteq \mathrm{mp}_1$.  Proof is trivial since $\mathrm{answer}(\mathrm{mp}_1)$ does not change.  Inductive step is $mathrm{mp}_n \subseteq \mathrm{mp}_n+1$.   Assume $\mathrm{mp}_n$ give the correct answer.  Argue on the lattice and the operational semantics for each rule that touches the status.  You have to reference in the proof every rule that updates the status!
\end{itemize}



