\newcommand{\epsnd}{\textit{Pnd\_s}}
\newcommand{\eprcv}{\textit{Pnd\_r}}
\newcommand{\qset}{\textit{Q\_s}}
\newcommand{\snd}{\textit{snd}}
\newcommand{\rcv}{\textit{rcv}}
\newcommand{\q}{\textit{q}}
\newcommand{\frm}{\textit{frm}}
\newcommand{\vbot}{\textit{v$-\bot$}}
\newcommand{\status}{\textit{s}}
\newcommand{\last}{\textit{l}}
\newcommand{\error}{\ensuremath{\mathbf{error}}}
\newcommand{\aidmap}{\textit{A}}
\newcommand{\thread}{\ensuremath{\mathit{t}}}
\newcommand{\aid}{\ensuremath{\mathit{a}}}
\newcommand{\ploc}{\ensuremath{\rho}}
\newcommand{\loc}{\ensuremath{\mathit{l}}}
\newcommand{\cmd}{\ensuremath{\mathit{c}}}
\newcommand{\op}{\ensuremath{\mathit{op}}}
\newcommand{\applyop}{\ensuremath{\mathrm{op}}}
\newcommand{\wait}{\ensuremath{\mathbf{wait}}}
\newcommand{\sendi}{\ensuremath{\mathbf{sndi}}}
\newcommand{\recvi}{\ensuremath{\mathbf{rcvi}}}
\newcommand{\traceentry}{\ensuremath{\sigma}}
\newcommand{\movelist}{\ensuremath{\delta}}
\newcommand{\assume}{\ensuremath{\mathit{assume}}}
\newcommand{\checkassume}{\ensuremath{\mathrm{assume}}}
\newcommand{\assert}{\ensuremath{\mathit{assert}}}
\newcommand{\checkassert}{\ensuremath{\mathrm{assert}}}
\newcommand{\ep}{\ensuremath{\mathbf{\gamma}}}
\newcommand{\src}{\ensuremath{\alpha}}
\newcommand{\dst}{\ensuremath{\beta}}
\newcommand{\true}{\ensuremath{\mathbf{true}}}
\newcommand{\false}{\ensuremath{\mathbf{false}}}
\newcommand{\match}{\ensuremath{\mathrm{m}}}
\newcommand{\matchl}{\ensuremath{\mathrm{match}}}
\newcommand{\findrecv}{\ensuremath{\mathrm{search_r}}}
\newcommand{\smt}{\ensuremath{\mathit{smt}}}


\newcommand{\addsend}{\ensuremath{\mathrm{addsend}}}
\newcommand{\getsend}{\ensuremath{\mathrm{getsend}}}
\newcommand{\removesend}{\ensuremath{\mathrm{removesend}}}
\newcommand{\addrecv}{\ensuremath{\mathrm{addrecv}}}
\newcommand{\getrecv}{\ensuremath{\mathrm{getrecv}}}
\newcommand{\removerecv}{\ensuremath{\mathrm{removefrecv}}}
\newcommand{\hextend}{\ensuremath{\mathrm{hextend}}}
\newcommand{\hlookup}{\ensuremath{\mathrm{hlookup}}}
\newcommand{\etalookup}{\ensuremath{\mathrm{etalookup}}}
\newcommand{\checkmatch}{\ensuremath{\mathrm{checkmatch}}}
\newcommand{\getep}{\ensuremath{\mathrm{getEP}}}
\newcommand{\getmarkremove}{\ensuremath{\mathrm{getMarkRemove}}}
\newcommand{\getlastsendreplace}{\ensuremath{\mathrm{getlastsend/replace}}}
\newcommand{\statuschange}{\left\{ \begin{array}{ll}  \status &\ \mathrm{if}\ |\epsnd(\dst)(\src)|>0\\
   \mathit{error} &\  \mathrm{otherwise}\end{array}\right .}

\newcommand{\reduce}[1]{\ensuremath{\rightarrow_{#1}}}
% Multiple reductions
\newcommand{\reduceK}[1]{\ensuremath{\rightarrow_{#1}^{*}}}
% Non-deterministic reductions
\newcommand{\reduceN}[1]{\ensuremath{\dashrightarrow_{#1}}}
\newcommand{\reduceNK}[1]{\ensuremath{\dashrightarrow_{#1}^{*}}}
\newcommand{\mt}{\ensuremath{\emptyset}}
\newcommand{\trace}{\ensuremath{\mathit{trace}}}
\newcommand{\movebot}{\ensuremath{\mathit{m}}}
\newcommand{\ret}{\ensuremath{\mathbf{ret}}}

\newsavebox{\boxLangSyntax}
\newsavebox{\boxEvalSyntax}

\begin{lrbox}{\boxLangSyntax}
\begin{minipage}[c]{0.4\linewidth}
\cfgstart
%  (ctp (thread ...))
\cfgrule{ctp}{\lp\cfgnt{\thread}$~\ldots$\rp}
%  (thread ([ploc cmd] ... ��))
\cfgrule{\thread}{\lp\lb\cfgt{\ploc}~\cfgnt{\cmd}\rb~$\ldots$~$\bot$\rp}
%  (cmd (assume e)
%       (assert e)
%       (x := e)
%       (sendi aid src dst e ploc number)
%       (recvi aid dst x ploc)
%       (wait aid)
%       )
\cfgrule{\cmd}{\lp\cfgt{assume}~\cfgnt{e}\rp}
   \cfgorline{\lp\cfgt{assert}~\cfgnt{e}\rp}
   \cfgorline{\lp\cfgt{x}~$:=$~\cfgnt{e}\rp}
   \cfgorline{\lp\cfgt{\sendi}~\cfgt{\aid}~\cfgnt{\src}~\cfgnt{\dst}~\cfgnt{e}~\cfgnt{\ploc}\rp}
   \cfgorline{\lp\cfgt{\recvi}~\cfgt{\aid}~\cfgnt{\dst}~\cfgt{x}~\cfgnt{\ploc}\rp}
   \cfgorline{\lp\cfgt{\wait}~\cfgt{\aid}\rp}
%  (move-�� ��
%            move-list)
\cfgrule{\movebot}{$\bot$\
                \cfgor\cfgt{\movelist}}
%  (move-list ((dst src) ... ��))
\cfgrule{\movelist}{\lp\lp\cfgt{\dst}~\cfgt{\src}\rp~\ldots~$\bot$\rp}
%  (trace (trace-entry ...))
\cfgrule{\trace}{\lp\cfgnt{\traceentry}~$\ldots$\rp}
% (trace-entry (ploc move-��))
\cfgrule{\traceentry}{\lp\cfgt{\ploc}~\cfgt{\movebot}\rp}
%  (e (op e e)
%     cmd
%     x
%     v)
\cfgrule{e}{\lp\cfgt{op}~\cfgnt{e}~\cfgnt{e}\rp}
   \cfgorline{\cfgnt{\cmd}}
   \cfgorline{\cfgt{x}}
   \cfgorline{\cfgnt{v}}
%  (v number
%     bool)
\cfgrule{v}{\cfgt{number}}
   \cfgorline{\cfgnt{bool}}
%  (bool true
%        false)
\cfgrule{bool}{\cfgt{true}}
   \cfgorline{\cfgt{false}}
%  (src ep)
%  (dst ep)
\cfgrule{\src}{\cfgt{\ep}}
\cfgrule{\dst}{\cfgt{\ep}}
%% Omitted as not terribly important
%  (ploc id)
%  (ep number)
%  (aid id)
%  (x id)
%  (id variable-not-otherwise-mentioned)
%  (loc number)
\cfgend
\end{minipage}
\end{lrbox}

\begin{lrbox}{\boxEvalSyntax}
\begin{minipage}[c]{0.4\linewidth}
%% TODO: define all the functions separately from the syntax so snd, frm, S, rcv, R, h, \eta
%% are actual functions rather than their current definition in the syntax.  Will simplify life.
\cfgstart
% (machine-state (h eta aid-map pending-s pending-r q-set ctp trace status smt))
\cfgrule{mstate}{\lp\cfgnt{h}\ \cfgnt{\ensuremath{\eta}}\ \cfgnt{\aidmap}\
  \cfgnt{\epsnd}\ \cfgnt{\eprcv}\ \cfgnt{\qset}\ \cfgnt{ctp}\ \cfgnt{\trace} \cfgnt{\status}\rp}
% (queue-state (pending-s q-set move-�� status))
\cfgrule{qstate}{\lp\cfgnt{\epsnd}\ \cfgnt{\qset}\ \cfgnt{\movebot}\ \cfgnt{\status}\rp}
%  (expr-state (h eta aid-map pending-s pending-r q-set e status k smt))
\cfgrule{estate}{\lp\cfgnt{h}\ \cfgnt{\ensuremath{\eta}}\ \cfgnt{\aidmap}\
  \cfgnt{\epsnd}\ \cfgnt{\eprcv}\ \cfgnt{\qset}\ \cfgnt{e} \cfgnt{\status} \cfgnt{k}\rp}
%  (h mt
%     (h [loc -> v]))
\cfgrule{h}{\cfgt{\mt}\
   \cfgor\lp\cfgnt{h}~\lb\cfgt{\loc}~$\rightarrow$~\cfgnt{v}\rb\rp}
%  (eta mt
%       (eta [x -> loc]))
\cfgrule{\ensuremath{\eta}}{\cfgt{\mt}\
    \cfgor\lp\cfgnt{\ensuremath{\eta}}~\lb\cfgt{x}~$\rightarrow$~\cfgt{\loc}\rb\rp}
%  (aid-map ((aid ep) ... ))
\cfgrule{\aidmap}{\lp\lp\cfgt{\aid}~\cfgt{\ep}\rp~$\ldots$\rp}
%  (pending-s mt
%                 (pending-s [dst -> from-set]))
\cfgrule{\epsnd}{\cfgt{\mt}\
   \cfgor\lp\cfgnt{\epsnd}~\lb\cfgnt{\dst}~$\rightarrow$~\cfgnt{frm}\rb\rp}
%  (from-set mt
%            (from-set [src -> send-set]))
\cfgrule{frm}{\cfgt{\mt}\
  \cfgor\lp\cfgnt{frm}~\lb\cfgnt{\src}~$\rightarrow$~\cfgnt{snd}\rb\rp}
%  (send-set mt
%            (send-set [aid -> v]))
\cfgrule{snd}{\lp\lb\cfgt{\aid}~\cfgt{v}\rb\ \ldots\rp}
%  (pending-r mt
%                 (pending-r [dst -> recv-set]))
\cfgrule{\eprcv}{\cfgt{\mt}\
   \cfgor\lp\cfgnt{\eprcv}~\lb\cfgnt{\dst}~$\rightarrow$~\cfgnt{rcv}\rb\rp}
%  (recv-set mt
%            (recv-set [aid -> x]))
\cfgrule{rcv}{\lp\lb\cfgt{\aid}~\cfgt{x}\rb\ \ldots\rp}
%  (q-set mt
%        (q-set [dst -> q]))
\cfgrule{\qset}{\cfgt{\mt}\
    \cfgor\lp\cfgnt{\qset}~\lb\cfgnt{\dst}~$\rightarrow$~\cfgnt{\q}\rb\rp}
%   (q ([aid -> v-��] ...))
\cfgrule{\q}{\lp\lb\cfgt{\aid}~\cfgt{\vbot}\rb~\ldots\rp}
%  (v-�� ��
%        v)
\cfgrule{\vbot}{\cfgt{$\bot$}\
            \cfgor\cfgt{v}}

%  (status status-temp
%          status-final)
%  (status-temp success-temp ;everything's fine
%          failure-temp ;assertion failed
%          infeasable-temp ;assumption failed
%          error-temp) ;impossible to execute
%  (status-final success-final
%                failure-final
%                infeasable-final
%                error-final)
\cfgrule{\status}{\cfgt{success}}
   \cfgorline{\cfgt{failure}}
   \cfgorline{\cfgt{infeasible}}
   \cfgorline{\cfgt{\error}}
%  (machine-state (h eta aid-map pending-s pending-r ctp trace status smt))
%  (expr-state (h eta aid-map pending-s pending-r e status k smt))
%(k ret
%     (assert * -> k)
%     (assume * -> k)
%     (x := * -> k)
%     (op * e -> k)
%     (op v * -> k)
%     )
\cfgrule{k}{\cfgt{ret}}
   \cfgorline{\lp\cfgt{assert}~\cfgt{*}~$\rightarrow$~\cfgnt{k}\rp}
   \cfgorline{\lp\cfgt{assume}~\cfgt{*}~$\rightarrow$~\cfgnt{k}\rp}
   \cfgorline{\lp\cfgt{x}~$:=$~\cfgt{*}~$\rightarrow$~\cfgnt{k}\rp}
   \cfgorline{\lp\cfgt{op}~\cfgt{*}~\cfgnt{e}~$\rightarrow$~\cfgnt{k}\rp}
   \cfgorline{\lp\cfgt{op}~\cfgt{v}~\cfgt{*}~$\rightarrow$~\cfgnt{k}\rp}
%     (match [aid -> src dst v] [aid -> dst x] -> k)
%     )
%  (smt (defs asrts)
%       )
%  (defs (any ...))
%  (asrts (any ...))
\cfgend
\end{minipage}
\end{lrbox}

\begin{figure}
\begin{center}
\setlength{\tabcolsep}{15pt}
\begin{tabular}{cc}
\scalebox{0.7}{\usebox{\boxLangSyntax}}
&
\scalebox{0.7}{\usebox{\boxEvalSyntax}}
\\ \\
(a) & (b)
\end{tabular}
\end{center}
\caption{The trace language syntax with its evaluation syntax for the operational semantics--bold face indicates a terminal. (a) The input syntax with terminals \textbf{x}, \textbf{\ploc} (which is unique), and \textbf{\aid} defined as strings and \textbf{\ep} as a number. (b) The evaluation syntax with terminal \textbf{\loc} defined as a number.}
\label{fig:expr:stx}
\end{figure}

\begin{figure}
\mprset{flushleft}
\scalebox{0.75}{
\begin{mathpar}
%(--> (h eta aid-map pending-s pending-r q-set
%           (thread_0 ...
%            ([ploc_0 cmd_0] [ploc_1 cmd_1] ... ��)
%            thread_2 ...)
%           ([ploc_0 move-��] trace-entry_1 ...) status smt)
%        (h_pr eta_pr aid-map_pr pending-s_prpr pending-r_pr q-set_prpr
%              (thread_0 ... ([ploc_1 cmd_1] ... ��) thread_2 ...)
%              (trace-entry_1 ...) status_prpr smt_prpr)
%        "Machine Step for queue movement"
%        (where (pending-s_pr q-set_pr move-��_pr status_pr)
%               ,(apply-reduction-relation** queue-reductions
%                                            (term (pending-s q-set move-�� status))))
%       ; "Machine Step"
%        (where (h_pr eta_pr aid-map_pr pending-s_prpr pending-r_pr q-set_prpr
%                     e status_prpr ret smt_pr)
%               ,(apply-reduction-relation** expr-reductions
%                                            (term (h eta aid-map pending-s_pr
%                                                     pending-r q-set_pr cmd_0
%                                                     status_pr ret smt))))
%        (where smt_prpr (add-po smt_pr ploc_0 ploc_1 ...)))
%
%   ))
\inferrule[Machine Step]{
  (\epsnd~\qset~\movebot~\status)~\reduceK{q}~
  (\epsnd_\mathit{p0}~\qset_\mathit{p0}~\movebot_p~\status_\mathit{p0}) \\\\
  (h~\eta~\aidmap~\epsnd_\mathit{p0}~\eprcv~\qset_\mathit{p0}~\cmd_0~\status_\mathit{p0}~\ret)~\reduceK{e}~
  (h_p~\eta_p~\aidmap_p~\epsnd_\mathit{p1}~\eprcv_p~\qset_\mathit{p1}~e~\status_\mathit{p1}~\ret)
}{
  (h~\eta~\aidmap~\epsnd~\eprcv~\qset~
           (\thread_0~\ldots~
            ([\ploc_0~\cmd_0]~[\ploc_1~\cmd_1]~[\ploc_2~\cmd_2]~\ldots)~\thread_2~\ldots)~
           ([\ploc_0~\movebot]~\traceentry_1~\ldots)~\status) \reduce{m} \\\\
  (h_p~\eta_p~\aidmap_p~\epsnd_\mathit{p1}~\eprcv_p~\qset_\mathit{p1}~
              (\thread_0~\ldots~([\ploc_1~\cmd_1]~[\ploc_2~\cmd_2]~\dots)~\thread_2~\dots)~
              (\traceentry_1~\dots)~\status_\mathit{p1})
}
\end{mathpar}}
\caption{Machine Reductions ($\reduce{m}$)}
\label{fig:machine}
\end{figure}

\begin{figure}
\mprset{flushleft}
\scalebox{0.75}{
\begin{mathpar}
%(define queue-reductions
%  (reduction-relation
%   lang
%   #:domain queue-state

%   (--> (pending-s q-set ((dst src) (dst_0 src_0) ... ��) status)
%        (pending-s_pr q-set_pr ((dst_0 src_0) ... ��) status_pr)
%        "Process queue movement"
%        (where (pending-s_pr aid v status_pr) (remove-send src dst pending-s status))
%        (where q-set_pr (add-send-pr dst aid v q-set status_pr)))
%
%))
\inferrule[Process Queue Movement]{
  ([\aid_s\ v]\ [\aid_1\ v_1]\ \ldots) = \epsnd(\dst)(\src) \\
   \epsnd_p = [\epsnd \mid \dst \mapsto [\epsnd(\dst) \mid \src \mapsto ([\aid_1\ v_1]\ \ldots)]] \\
   ([\aid_1\ v_1]\ \ldots) = \qset(\dst) \\
   \qset_p = [\qset \mid \dst \mapsto ([\aid_s\ v]\ [\aid_1\ v_1]\ \ldots)] \\
   \status_p = \statuschange}{
  (\epsnd\ \qset\ ((\dst\ \src)\ (\dst_0\ \src_0)\ \ldots\ \bot)\ \status) \reduce{q}(\epsnd_p\ \qset_p\ ((\dst_0\ \src_0)\ \ldots\ \bot)\ \status_p)
}
\end{mathpar}}
\caption{Queue Reductions ($\reduce{q}$)}
\label{fig:queue}
\end{figure}

\begin{figure}
\scalebox{0.75}{%
\mprset{flushleft}
\begin{mathpar}
%(--> (h eta aid-map pending-s pending-r q-set  x status k smt)
%        (h eta aid-map pending-s pending-r q-set v status k smt)
%        "Lookup Variable"
%        (where v (h-lookup h (eta-lookup eta x))))
\inferrule[Variable Lookup]{}{
  (h\ \eta\ \aidmap\ \epsnd\ \eprcv\ \qset\ x\ \status\ k)\ \reduce{e} \\\\
  (h\ \eta\ \aidmap\ \epsnd\ \eprcv\ \qset\  h(\eta(x))\ \status\ k )
}
\hspace*{60pt}
%   (--> (h eta aid-map pending-s pending-r q-set (op e_0 e) status k smt)
%        (h eta aid-map pending-s pending-r q-set e_0 status (op * e -> k) smt)
%        "Expr l-operand")
\inferrule[Left Operand]{}{
  (h\ \eta\ \aidmap\ \epsnd\ \eprcv\ \qset\ (\op\ e_0\ e)\ \status\ k) \reduce{e} \\\\
  (h\ \eta\ \aidmap\ \epsnd\ \eprcv\ \qset\ e_0\ \status\ (\op\ *\ e \rightarrow k))
}
\and
%   (--> (h eta aid-map  pending-s pending-r q-set v status (op * e -> k) smt)
%        (h eta aid-map  pending-s pending-r q-set e status (op v * -> k) smt)
%        "Expr r-operand")
\inferrule[Right Operand]{}{
  (h\ \eta\ \aidmap\ \epsnd\ \eprcv\ \qset\ v\ \status\ (\op\ *\ e \rightarrow k)) \reduce{e} \\\\
  (h\ \eta\ \aidmap\ \epsnd\ \eprcv\ \qset\ e\ \status\ (\op\ v\ *\ \rightarrow k))
}
\hspace*{20pt}
%   (--> (h eta aid-map pending-s pending-r q-set v_r status (op v_l * -> k) smt)
%        (h eta aid-map pending-s pending-r q-set v_res status k smt)
%        "Binary Operation Eval"
%        (where v_res (apply-op op v_l v_r)))
\inferrule[Binary Operation]{}{
  (h\ \eta\ \aidmap\ \epsnd\ \eprcv\ \qset\ v_r\ \status\ (\op\ v_l\ * \rightarrow k)) \reduce{e} \\\\
  (h\ \eta\ \aidmap\ \epsnd\ \eprcv\ \qset\ \applyop(v_l, v_r)\ \status\ k)
}
\and
%   (--> (h eta aid-map pending-s pending-r q-set (assume e) status k
%           (defs (any ...)))
%        (h eta aid-map pending-s pending-r q-set e status (assume * -> k )
%           (defs (e any ...)))
%        "Assume Pull Out Expr")
\inferrule[Assume Expressions Evaluation]{}{
  (h\ \eta\ \aidmap\ \epsnd\ \eprcv\ \qset\ (\assume\ e)\ \status\ k) \reduce{e} \\\\
  (h\ \eta\ \aidmap\ \epsnd\ \eprcv\ \qset\ e\ \status\ (\assume\ * \rightarrow k))
}
\hspace*{22pt}
%   (--> (h eta aid-map pending-s pending-r q-set v status (assume * -> k) smt)
%        (h eta aid-map pending-s pending-r q-set v status_pr k smt)
%        "Assume Cmd"
%        (where status_pr (check-assume v status)))
\inferrule[Assume Command]{}{
  (h\ \eta\ \aidmap\ \epsnd\ \eprcv\ \qset\ v\ \status\ (\assume\ * \rightarrow k)) \reduce{e} \\\\
  (h\ \eta\ \aidmap\ \epsnd\ \eprcv\ \qset\ v\ \checkassume(v, s)\ k)
}
\and
%   ;;Negate expression and add it to the SMT assertions.
%   (--> (h eta aid-map pending-s pending-r q-set (assert e) status k
%           (defs (any ...)))
%        (h eta aid-map pending-s pending-r q-set e status (assert * -> k)
%           (defs ((not e) any ...)))
%        "Assert Pull Out Expr")
\inferrule[Assert Expressions Evaluation]{}{
  (h\ \eta\ \aidmap\ \epsnd\ \eprcv\ \qset\ (\assert\ e)\ \status\ k) \reduce{e} \\\\
  (h\ \eta\ \aidmap\ \epsnd\ \eprcv\ \qset\ e\ \status\ (\assert\ * \rightarrow k))
}
\hspace*{22pt}
%   (--> (h eta aid-map pending-s pending-r q-set v status (assert * -> k) smt)
%        (h eta aid-map pending-s pending-r q-set v status_pr k smt)
%        "Assert Eval"
%        (where status_pr (check-assert v status)))
\inferrule[Assert Command]{}{
  (h\ \eta\ \aidmap\ \epsnd\ \eprcv\ \qset\ v\ \status\ (\assert\ * \rightarrow k)) \reduce{e} \\\\
  (h\ \eta\ \aidmap\ \epsnd\ \eprcv\ \qset\ v\ \checkassert(v, s)\ k)
}
\and
%   (--> (h eta aid-map pending-s pending-r q-set (x := e) status k
%           (defs (any ...)))
%        (h eta aid-map pending-s pending-r q-set e status (x := * -> k)
%           (defs ((= x e) any ...)))
%        "Assign Pull Out Expr")
\inferrule[Assign Expressions Evaluation]{}{
  (h\ \eta\ \aidmap\ \epsnd\ \eprcv\ \qset\ (x\ :=\ e)\ \status\ k) \reduce{e} \\\\
  (h\ \eta\ \aidmap\ \epsnd\ \eprcv\ \qset\ e\ \status\ (x\ :=\ * \rightarrow k))
}
\hspace*{22pt}
%   (--> (h eta aid-map pending-s pending-r q-set v status (x := * -> k) smt)
%        (h_pr eta aid-map pending-s pending-r q-set v status k smt)
%        "Assign Expr"
%        (where h_pr (h-extend* h [(eta-lookup eta x) -> v])))
\inferrule[Assign Command]{}{
  (h\ \eta\ \aidmap\ \epsnd\ \eprcv\ \qset\ v\ \status\ (x\ :=\ *\ \rightarrow k)) \reduce{e} \\\\
  ([h \mid \eta(x) \mapsto v]\ \eta\ \aidmap\ \epsnd\ \eprcv\ \qset\ v\ \status\ k)
}
\and
%   ;;Adds sendi cmd to aid-map and pending-s. - VALIDATED
%   (--> (h eta ([aid_x ep_x] ...) pending-s pending-r q-set
%           (sendi aid src dst x ploc number) status k ((any_d ...) (any_a ...)))
%        (h eta ([aid src] [aid_x ep_x] ...) pending-s_pr pending-r q-set
%           true status k ((any_0 any_d ...) (any_1 any_a ...)))
%        "Sendi Cmd x -> v"
%        (where v (h-lookup h (eta-lookup eta x)))
%        (where any_0 (define aid :: send))
%        (where any_1 (make-send aid src dst x ploc number))
%        (where pending-s_pr (add-send pending-s [aid -> src dst v])))
\inferrule[Sndi Command]{
  ([\aid_1\ v_1]\ \ldots) = \epsnd(\dst)(\src) \\ \epsnd_p = [\epsnd \mid \dst \mapsto [\epsnd(\dst) \mid \src \mapsto ([\aid_0\ h(\eta(x))]\ [\aid_1\ v_1]\ \ldots)]]
}{
  (h\ \eta\ ([\aid\ \ep]\ \ldots)\ \epsnd\ \eprcv\ \qset\
     (\sendi\ \aid_0\ \src\ \dst\ x)\ \status\ k) \reduce{e}
  (h\ \eta\ ([\aid_0\ \src]\ [\aid\ \ep]\ \ldots)\ \epsnd_p\ \eprcv\ \qset\ \true\ \status\ k)
}
\and
%   ;;Adds recvi cmd to aid-map and pending-r. - VALIDATED
%   (--> (h eta ([aid_x ep_x] ...) pending-s pending-r q-set
%           (recvi aid dst x ploc) status k ((any_d ...) (any_a ...)))
%        (h eta ([aid dst] [aid_x ep_x] ...) pending-s pending-r_pr q-set
%           true status k ((any_0 any_d ...) (any_1 any_a ...)))
%        "Recvi Cmd0"
%        (where pending-r_pr (add-recv pending-r [aid -> dst x]))
%        (where any_0 (define aid :: recv))
%        (where any_1 (make-recv aid dst x ploc)))
\inferrule[Rcvi Command]{
  ([\aid_1\ x_1]\ \ldots) = \eprcv(\dst) \\
   \eprcv_p = [\eprcv \mid \dst \mapsto ([\aid_0\ x_0]\ [\aid_1\ x_1]\ \ldots)]
}{
  (h\ \eta\ ([\aid\ \ep]\ \ldots)\ \epsnd\ \eprcv\ \qset\ (\recvi\ \aid_0\ \dst\ x_0)\ \status\ k) \reduce{e}
  (h\ \eta\ ([\aid_0\ \dst]\ [\aid\ \ep]\ \ldots)\ \epsnd\ \eprcv_p\ \qset\ \true\ \status\ k)
}
\and
%(--> (h eta aid-map pending-s pending-r q-set (wait aid) status k smt)
%        (h eta aid-map pending-s pending-r q-set true status k smt)
%        "Sendi Wait Cmd"
%        (where false (find-recv pending-r aid) ))
\inferrule[Wait (sndi) Command]{}{
  (h\ \eta\ \aidmap\ \epsnd\ \eprcv\ \qset\ (\wait\ \aid_s)\ \status\ k) \reduce{e}
  (h\ \eta\ \aidmap\ \epsnd\ \eprcv\ \qset\ true\ \status\ k)
}
\and
   %(--> (h eta aid-map pending-s pending-r q-set (wait aid_r)
%           status k (defs (any_a ...)))
%        (h_pr eta aid-map_pr pending-s pending-r_pr q-set_pr true
%              status_pr k
%              (defs ((MATCH aid_r aid_s) any_a ...)))
%        "Recvi Wait Cmd"
%            ;; get recv and corresponding send command, mark each send in front of the destination send (by set variable "mark" to 1)
%        (where true (find-recv pending-r aid_r) )
%        (where (aid-map_0 dst_r) (get-ep aid-map aid_r))
%        (where (h_pr aid_s pending-r_pr q-set_pr status_pr)
%               (get-mark-remove h eta pending-r q-set dst_r aid_r status))
%        (where (aid-map_pr src_s) (get-ep aid-map_0 aid_s))
%        )
%   ))
\inferrule[Wait (rcvi) Command]{
    ([\aid_0\ \ep_0]\ \ldots\ [\aid_s\ \src]\ \ldots\ [\aid_r\ \dst]\ \ldots\ [\aid_1\ \ep_1]) = \aidmap\\
    %\aidmap_0 = ([\aid_0\ \ep_0]\ \ldots\ [\aid_1\ \ep_1])\\
    (h_p\ \aid_s\ \eprcv_p\ \qset_p\ \status_p) = \getmarkremove(h, \eta, \eprcv, \qset, \dst, \aid_r, \status) \\
    %([\aid_0\ \ep_0]\ \ldots\ [\aid_s\ \src][\aid_1\ \ep_1]) = \aidmap_0\\
    \aidmap_p = ([\aid_0\ \ep_0]\ \ldots\ [\aid_1\ \ep_1])
}{
    (h\ \eta\ \aidmap\ \epsnd\ \eprcv\ \qset\ (\wait\ \aid_r)\ \status\ k) \reduce{e}
    (h_p\ \eta\ \aidmap_p\ \epsnd\ \eprcv_p\ \qset_p\ true\ \status_p\ k)
}

\end{mathpar}}
\caption{Expression Reductions ($\reduce{e}$)}
\label{fig:expression}
\end{figure}

\begin{comment}
\begin{figure}
\begin{center}
\begin{lstlisting}
(((((mt [1 -> 0]) [2 -> 0]) [3 -> 0])[4 -> 4])
((((mt [A -> 1]) [B -> 2]) [C -> 3])[e -> 4])
()
mt
mt
(
(
[t0_0 (sendi send1 1 0 e t0_0 1)]
[t0_1 (wait send1)]
)
(
[t1_0 (recvi recvA 0 A t1_0)]
[t1_1 (wait recvA)]
[t1_2 (assert (= 4 A))]
)

)
(t0_0 t0_1 t1_0 [t1_1 send1] t1_2)
success-temp
(()()))
))
\end{lstlisting}
\end{center}
\caption{A trace language program to send and receive a single message.}
\label{fig:prog}
\end{figure}
\end{comment}

\section{Trace Language}\label{sec:trace}
The trace language is the theoretical framework for the match-pair
encoding.  The language syntax describes a CTP with a single execution
trace on the same CTP.  The evaluation syntax with its operational
semantics define how to execute the CTP, following the specified
trace, and define when that execution is a success (causes no assertion
violation), a failure (causes an assertion violation), infeasible
(causes an assume to not hold), or an error (uses a bogus match pair).  \secref{sec:smt} defines the encoding of a trace language program as an
SMT problem and extends that encoding to capture a set of possible
traces using match-pairs.

\subsection{Syntax}
\figref{fig:expr:stx}(a) is the syntax for a trace language program.
This presentation uses ellipses ($\ldots$) to represent zero or more
repetitions, bold-face to indicate terminals, and omits commas in tuples for cleanliness. A trace language
program is a CTP with a trace defining a sequential run of the CTP.
The language defines a CTP (\textit{ctp}) as a list of threads.  A
thread (\thread) is a list of pairs with each pair being a program
location and a command.  For simplicity, commands (\textit{c}) are
restricted to assume, assert, assignment, non-blocking send (\snd) and
receive (\rcv), and wait. We model a trace as an order of executed locations identified by (\ploc) and a series of queue movements that move a message from a source queue to a destination queue. The queue movement (\movebot) is either a special symbol indicating no movement or a list of move commands.
The move list (\movelist) consists of several pairs of end-points where the first and second end-points are the destination and source end-points respectively.
Each pair in the move list refers to a queue movement such that the first pending send of the source end-point should be moved to the send queue of the destination end-point.
The non-terminal \aid\ in the grammar is a
unique string identifier associated with a send or receive command
referred to as an action ID in the text.  The wait command takes a
single action ID belonging to the associated send or receive action. The non-terminals $\src$ and $\dst$ are
source and destination end-points respectively.  The non-terminal $\ep$
is a number.  The terminal \textbf{x} is any string not mentioned in
the grammar definition and represents a program variable.    Expressions (\textit{e}) are defined
using prefix notation over binary operators.  The bottom of
\figref{fig:mcapi} is an example \textit{ctp} in the trace language
(omitting the first and second columns and trivially grouping each
thread and its commands into an appropriate list using parenthesis).

A sequential trace of a CTP in the grammar is a list of trace entries
(\traceentry).  A trace entry is a pair consisting of a program location (\ploc) and either a queue movement list (\movebot) or a symbol ($\bot$). An example
of a trace can be seen in the bottom of \figref{fig:mcapi} in the
second column by following the sequential order in the first column
which starts on program location \texttt{2\_0} of task 2.  Notice that whenever the
trace reaches a wait command on a receive action, the trace includes
a queue movement list where the destination is listed followed by the source. Messages are delivered from the source queue to the destination queue. In other words, the trace resolves any non-determinism in scheduling or
message buffering that is present in the CTP.

\subsection{Operational Semantics}
The operational semantics for the trace language are given by a term
rewriting system using a \textit{CESK} style machine
\footnote{The \textit{CESK} machine state is represented with a \textbf{C}ontrol string,
  \textbf{E}nvironment, \textbf{S}tore, and \textbf{K}ontinuation.}
only the machine is augmented to include additional structure for
modeling message passing. \figref{fig:expr:stx}(b) defines the machine
state and other syntax relating to evaluation.

A machine step ($\reduce{m}$) in \figref{fig:machine} moves a thread forward by a
single command.  The rules operate on a machine state tuple
(\textit{mstate}) defined in \figref{fig:expr:stx}(b).  The tuple
can be partitioned into members relating to the \textit{CESK}
machine, members relating to the message passing model, and the trace status.  The CESK machine members are
\textit{ctp} (the list of thread command sequences), $\eta$ (an environment mapping a variable $x$ to a location $l$),
$h$ (a store mapping a location $l$ to a value $v$), and $k$ (a continuation).
Among the members of the message passing model,
\aidmap\ is a dictionary mapping an action ID \aid\ to an end-point \ep.
\epsnd\ is a set of send queues where each queue is uniquely identified first by the source end-point and then by the destination end-point. The queue itself holds pairs consisting of an action ID and value (\aid,v).
\eprcv\ is a set of receive queues where each queue is uniquely identified by the destination end-point. The queue itself holds pairs consisting of an action ID and a variable (\aid,$x$). \qset\ is also a set of queues where each queue is uniquely identified by the destination end-point. The queue itself holds pairs consisting of an action ID \aid\ and either a value $v$ or a symbol $\bot$. Intuitively, \epsnd\, \qset\ and \eprcv\ are end-point queues tracking actions with associated values (sends) or variables (receives). Both \epsnd\ and \qset\ store the sends. \qset\ holds delivered messages that are ready to be received. A message moves from \epsnd\ to \qset\ through queue movement lists in the trace on the CTP.

The trace status in the \textit{mstate} nine-tuple is given by
\status\ which ranges over a lattice:
\[\cfgt{success} \prec \cfgt{failure} \prec \cfgt{infeasible} \prec \cfgt{\error}\]
The trace status only moves monotonically up the lattice starting from
success.  A success trace completes the entire trace, meets all the
assume statements, and does not fail an assertion.  A failure trace
completes the entire trace, meets all the assume statements, but fails
an assertion.  An infeasible trace completes the entire trace but does
not meet all the assume statements.  An error is a trace that does
not complete.

We have several CESK machines that handle different aspects of the CTP and trace. The machine step moves one step on the trace. In each machine step, the queue machine processes any queue movements in that step, and the expression machine handles any expressions in the command associated with the trace step.

The \textit{Machine Step} inference in \figref{fig:machine} matches any \textit{mstate} that has a thread whose first list entry matches the program location in the head of the trace.  A match on the inference
rewrites the \textit{mstate} with new entries for each member of the
nine-tuple by first applying the queue reduction relation until no more reductions apply (all queue movements are processed in the trace entry first) and then applying the expression reduction relation until no
more reductions apply (as indicated by the asterisk). Note that queue reductions perform the queue movement such that all send actions are moved to the destination send queues. After completing all queue movements, messages can be delivered in the process of expression reduction.

The queue reduction for each command of the queue movement list in the trace entry is given in \figref{fig:queue}. The definition of the \textit{qstate} four-tuple is presented in the evaluation syntax in \figref{fig:expr:stx}(b). The symbol $\bot$ in the queue movement \movebot\ indicates that no more queue movement follows. A message moves from pending to delivered through queue movement lists. Each reduction step processes the first pair of the queue movement list and reduction steps follow until $\bot$ is shown.

Expression reductions for each command in the trace language are given
in \figref{fig:expression} and are defined over the \textit{estate}
nine-tuple in the evaluation syntax of \figref{fig:expr:stx}(b) which
includes a continuation $k$.  The \cfgt{ret} continuation indicates
that nothing follows, and an asterisk in a continuation is a place
holder indication where evaluation is taking place.  For example, the
\emph{Assume Expressions Evaluation} creates a continuation indicating
that it is first evaluating the expression in the assume command.
Once that expression reduces to a value, then the \emph{Assume
  Command} inference matches to check the assumptions.


The expression reductions use several helper functions. The function \emph{op}$(v_l,v_r)$
applies the ``op'' to the left and right operands. The function $\mathrm{getMarkRemove}$ is explained in the later paragraph. The other helper functions are defined below:
\begin{eqnarray*}
\checkassert(v,\status) &=& \left\{ \begin{array}{ll}
                                       \cfgt{failure} & \mathrm{if}~\status \prec \cfgt{failure}\ \wedge \\
                                                      & v = \false \\
                                             \status  & \mathrm{otherwise}
                                    \end{array} \right .\\
\checkassume(v,\status) &=& \left\{ \begin{array}{ll}
                                       \cfgt{infeasible} & \mathrm{if}~\status \prec \cfgt{infeasible}\ \wedge \\
                                                         & v = \false \\
                                                \status  & \mathrm{otherwise}
                                    \end{array} \right . \\
\end{eqnarray*}
Note that the status only moves monotonically up the lattice as
mentioned previously.  The notations $h(\eta(x))$ (Sndi Command), $\eprcv(\dst)$ (Rcvi Command), and
$\epsnd(\dst)(\src)$ (Sndi Command) in \figref{fig:expression} are used for lookup. For example, $\epsnd(\dst)(\src)$ returns a list of pairs of
action IDs and values as defined in \figref{fig:expr:stx}(b).

The \emph{Sndi Command} and \emph{Rcvi Command} in
\figref{fig:expression} update \eprcv\ and \epsnd\ respectively with
information to complete a message send or receive at a wait command.
Consider a portion of the \emph{Rcvi Command}:
\begin{eqnarray*}
\eprcv_p &=& [\eprcv \mid \dst \mapsto
  ([\aid_0\ x_0]\ [\aid_1\ x_1]\ \ldots)]\\
([\aid_1\ x_1]\ \ldots) &=& \eprcv(\dst)
\end{eqnarray*}
$\eprcv_p$\ is a new set, just
like the old set $\eprcv$, only the new set maps the destination
end-point $\dst$\ to its contents in the old set plus the added
entry to the front of the list of the action ID and variable for the
receive command being evaluated. Considering the entire rule in the figure, it also updates the
action ID map, $\aidmap = ([\aid_0\ \dst]\ [\aid\ \ep]\ \ldots)$, to include the coupling between the action ID
and its destination end-point.

The function of the \emph{Wait(SNDI) Command} rule in \figref{fig:expression} is to consume the wait command
only. The function of the other rule for the wait command is more involved. The \emph{Wait(RCVI) Command} rule is complicated because a CTP can wait on receive actions out of program order. Consider the following trace
 \[(\ldots\ (\mathit{rcvA}~\bot)\ (\mathit{rcvB}~\bot)\ (\mathit{wait(rcvB)}~\bot)\ (\mathit{wait(rcvA)}~\bot)\ \ldots)\]
omitting the regular definition for the commands and the trace. It is perfectly valid to now call $wait(rcvB)$ even though it appeared after the $rcvA$ action. The \emph{Wait(RCVI) Command} handles this very situation in function $\mathrm{getMarkRemove}$. %by first finding in the pending receive queue the subqueue from the head of the queue to the receive being waited on. The function then gets the send operation in the delivered queue \qset\ with the same length of the destination receive from the head of the queue. The length of the delivered queue \qset\ must be at least as long as this sub queue. The actual wait command then updates the indicated variable in every send preceding the one that matches the receive indicated by the wait command to reflect the completion of those commands which must precede the one indicated in the wait command.
Consider the same trace above, we have the following structures of the \emph{pending receive queue} and the \emph{delivered queue} respectively,
\[\eprcv:~(\ldots~(\dst~\rightarrow~([rcvA~x_0]~[rcvB~x_1]~\ldots))~\ldots)\]
\[\qset:~(\ldots~(\dst~\rightarrow~([snd1~v_0]~[snd2~v_1]~\ldots))~\ldots)\]
Function $\mathrm{getMarkRemove}$ in the command $wait(rcvB)$ works as the following steps: first finding the pair $[rcvB~x_1]$ in the pending receive queue; then getting the pair $[snd2~v_1]$ in the delivered queue; then marking the pair $[snd1~v_0]$ as first assigning $v_0$ to variable $x_0$ in the heap then updating the pair $[snd1~v_0]$ to $[snd1~\bot]$; finally removing the pair $[rcvB~x_1]$ and $[snd2~v_1]$ in $\eprcv$ and $\qset$, respectively.

The syntax with the operational semantics, as presented, are
implemented directly in PLT Redex.  PLT Redex is a language for
testing and debugging semantics using term rewriting and is part of the
Racket runtime.

The following definition is important to the proof of the match pair encoding in \secref{sec:smt}.
\begin{definition}
A machine state $(h\ \eta\ \aidmap\ \epsnd\ \eprcv\ \qset\ \cfgnt{ctp}\ \trace\ \status\ k)$ is well formed if and only if it reduces to a final
state where the CTP and trace run to completion, having matched every send and receive call, and the status is either success,
failure, or infeasible:
\[
\begin{array}{l}
(h\ \eta\ \aidmap\ \epsnd\ \eprcv\ \qset\ \cfgnt{ctp}\ \trace\ \status\ k) \reduceK{m} \\
(h_p\ \eta_p\ ()\ \epsnd_p\ \eprcv_p\ \qset_p\ (()\ ...)\ ()\ \status_p\ ret)
\end{array}
\]
such that
\[
\begin{array}{llll}
\forall \dst, \forall \src, \epsnd_p(\dst)(\src) = ()  & \wedge & \forall \dst, \eprcv_p(\dst) = () & \wedge \\
\forall \dst, \qset_p(\dst) = () & \wedge &  \status_p\ \prec \cfgt{error}
\end{array}
\]
For convenience, we define the function $\mathrm{status}(m)$ to return the final status after reduction of a well formed machine state $m$.
\end{definition}

Intuitively, a well-formed machine state completes all the transitions of send and receive calls, meaning that there are no elements in $\aidmap$, $\epsnd$, $\eprcv$ and $\qset$, the commands in the CTP and the execution trace are correctly executed, and the status of state never enters $\cfgt{error}$.

